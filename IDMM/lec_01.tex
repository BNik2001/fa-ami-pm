\documentclass[12pt,a4paper]{article}
\usepackage[14pt]{extsizes} 
\usepackage[utf8]{inputenc}
\usepackage{amsmath}
\usepackage{amsfonts}
\usepackage{amssymb}
\usepackage{cmap}
% for fonts
    \usepackage[T2A, T1]{fontenc}
    \usepackage[english, russian]{babel}
    \usepackage{fontspec}
    \defaultfontfeatures{Ligatures=TeX,Renderer=Basic}
    \setmainfont[Ligatures={TeX, Historic}]{Times New Roman}
    \setsansfont{Times New Roman}
    \setmonofont{Courier New}
%plot
\usepackage{pgf,tikz,pgfplots}
\pgfplotsset{compat=1.15}
\usepackage{mathrsfs}
\usetikzlibrary{arrows}
\pagestyle{empty}
%plot
\usepackage{float}% for \begin{figure}[H]
\usepackage{cases}
\usepackage{graphicx}
\usepackage[left=2cm,right=2cm,top=2cm,bottom=2cm]{geometry}
\author{GH-TIMe}
\begin{document}
\begin{center}
\section*{Лекция №1: Интерактивное принятие решений: игры и равновесия}
\end{center}

3 балла посещаемость, 17 баллов успеваемость (домашняя работа, активность, контрольная работа)

\begin{center}
\textbf{План}
\end{center}

\begin{enumerate}
\item Индивидуальное принятие решений;
\item Интерактивное принятие решений;
\item Бескоалиционная игра в нормальной форме;
\item Принципы оптимальности в бескоалиционных играх;
\end{enumerate}

Модель индивидуального рационального поведения 1
\begin{itemize}
\item Пусть агент (игрок) способен выбирать некоторое действие (стратегию) $x$ из множества $X$ допустимых действий;
\item В результате выбора действия $x \in X$ агент получает выйшгрыш $f(x)$, где $f: X \rightarrow R^1$ -- целевая функция (выйгрыш функция), отражающая предпочтения агента;
\item Выбор действия агентом определяется правилом индивидуального рационального выбора $P(f,X) \subseteq X$:
$$P(f,X) = Arg \max_{x \in X} f(x)$$
\end{itemize}

Модель рационального поведения 2:
\begin{itemize}
\item Пусть агент способен выбирать некоторые действие $x$ из множества $X$ допустимых действий с учётом неопределенного параметра $\theta \in \Theta$ -- состояние природы;
\item В результате выбора действий $x \in X$ и реализации состояния природы $\theta \in \Theta$ агент получает выйгрыш $f(\theta \in \Theta)$, где $f: \theta \times X \rightarrow R^1$ -- целевая функция, отражающая предпочтения агента.
\end{itemize}

Уровни информированности агента в условиях индивидульного выбора:
\begin{itemize}
\item Интервальная неопределённость (известно только множество $\theta$);
\item Вероятностная неопределённость (известно вероятностное распределение значений неопределённых параметров $\theta \in \Theta$);
\item Нечеткая неопределённость (известна функция принадлежности значений неопределённых параметров $\theta \in \Theta$) 
\end{itemize}
\begin{itemize}
\item Процедура устранения неопредедленности
$$f \underset{I}{\Rightarrow} \hat{f}$$
\item Выбор действия агентом определяется правилом индивидуального рационального выбора 
$$P(f,X,I) = Arg \max_{x \in X} \hat{f}(x)$$
\end{itemize}
Полная неопределённость устраняется принципом гарантированного результата при условии, что множество состояния природы образуют полную группу событий.

Интерактивное принятие решений -- это принятие решений в условии конфликта интересов многих сторон с непротивоположными интересами (с возможностью объединения в коалицию). При этом также могут учитываться разлицчные уровни информированности сторон.

\begin{center}
\textbf{Теоретико-игровая модель (бескоалиционная игра)}
\end{center}
Cистема:
$$\Gamma = (N,\{X_i\}_{i \in N}, \{H_i\}_{i \in N}),$$
$N = \{1, 2, \cdots, n\}$ -- множество игроков,\\
$X_i$ -- множество статегий игрока $i$;\\
$H_i$ -- Функция выигрыша игрока $i$, определённая на декартовом произведении множеств стратегий игроков $X =\prod_{i=1}^{n} = X_i$ (множество ситуаций игры)

Игроки одновременно и независимо друг от друга выбирают свои стратегии $x_i$ из множества стратегий $X_i$, $i = 1, 2, \cdots, n$, в результате формируется ситуация:
$$x = (x_1, x_2, \cdots, x_n),$$
$$x_i \in X_i$$
После этого каждый игрок $i$ получает выигрыш $H_i(x)$

\textbf{Случай двух игроков}

Игра двух лиц $\Gamma$ в нормальной форме определяется системой:
$$\Gamma = (x_1, x_2, H_1, H_2)$$
где $X_1$ -- множество стратегий первого игрока,\\
$X_2$ -- множество стратегий второго игрока\\
$X_1 \times X_2$ -- множество ситуаций игры,\\
$H_1: X_1 \times X_2 \rightarrow R^1, H_2: X_1 \times X_2 \rightarrow R^1$ -- функция выйгрыша игроков 1 и 2. Конечная бескоалиционная игра двух лиц называется биоматричной:

\begin{equation*}
H_1 = A =
\begin{bmatrix}
\alpha_{11} & \cdots & \alpha_{1n}\\
\cdots & \cdots & \cdots\\
\alpha_{m1} & \cdots & \alpha_{mn}
\end{bmatrix}
\end{equation*}

\begin{equation*}
H_2 = B =
\begin{bmatrix}
\beta_{11} & \cdots & \beta_{1n}\\
\cdots & \cdots & \cdots\\
\beta_{m1} & \cdots & \beta_{mn}
\end{bmatrix}
\end{equation*}

Для игры двух лиц: $\Gamma = (x_1, x_2, H_1, H_2)$ ситуация $(x_1^*, x_2^*)$ является равновестной по Нэшу, если неравенства:
$$H_1(x_1, x_2^*) \leq H_1(x_1^*, x_2^*)$$
$$H_2(x_1^*, x_2) \leq H_2(x_1^*, x_2^*)$$
выполняются для всех $x_1 \in X_1$, $x_2 \in X_2$.

Игровая ситуация в неантоганистической игре называется \textit{равновестной по Нэшу}, если не один из игроков в единоличном порядке не может её ищменить не ухудшив своего положения. Справедливо для бескоалиционной игры.

Ситуация $x$ с черточкой в бескоалиционной игре $\Gamma$ называется оптимальной по Парето, если не существует ситуации х принадлежит $X$, для которой справедливо

$$H_i(x) \geqslant H_i(\overline{x}), \forall i \in N$$
$$H_{i_0}(x) > H_{i_0}(\overline{x})$$ хотя бы для одного $i_0 \in N$

т.е. не существует другой ситуации х, которая была бы предпочтительней х с черточкой для всех игроков.

Игровая ситуация называется \textit{оптимальной по Парето}, если не один из игроков не может в единоличном порядке улучшить своё положение не ухудшив хотя бы одного из игроков.

\textbf{Примеры}

$(A, B) = $
\begin{tabular}{|c|c|c|}
\hline 
 & $\beta_1$ & $\beta_2$ \\ 
\hline 
$a_1$ & (5; 5) & (0; 10) \\ 
\hline 
$a_2$ & (10; 0) & (1; 1) \\ 
\hline 
\end{tabular}

(1, 1) оптимально по Нэша\\
(5, 5) оптимальная по Парето\\

$(A, B) = $
\begin{tabular}{|c|c|c|}
\hline 
 & $\beta_1$ & $\beta_2$ \\ 
\hline 
$a_1$ & (4; 1) & (0; 0) \\ 
\hline
$a_2$ & (0; 0) & (1; 4) \\ 
\hline 
\end{tabular} 




\end{document}