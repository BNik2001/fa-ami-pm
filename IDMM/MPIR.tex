\documentclass[12pt,a4paper]{article}
\usepackage[14pt]{extsizes} 
\usepackage[utf8]{inputenc}
\usepackage{amsmath}
\usepackage{amsfonts}
\usepackage{amssymb}
\usepackage{cmap}
\usepackage{float}
\usepackage[table]{xcolor}
% for fonts
    \usepackage[T2A, T1]{fontenc}
    \usepackage[english, russian]{babel}
    \usepackage{fontspec}
    \defaultfontfeatures{Ligatures=TeX,Renderer=Basic}
    \setmainfont[Ligatures={TeX, Historic}]{Times New Roman}
    \setsansfont{Times New Roman}
    \setmonofont{Courier New}
\usepackage{pgfplots} % plot
\usepackage{cases}
\pagestyle{empty} %  выключаенм нумерацию
\pgfplotsset{compat=1.15}
\usepackage{graphicx}
\usepackage[left=2cm,right=2cm,top=2cm,bottom=2cm]{geometry}
\author{GH-TIMe}
\begin{document}

\begin{center}
\textbf{Домашнее задание № 1}
\end{center}
\begin{center}
\textbf{Теоретико-игровой анализ проблемы перевода капитала в офшорные зоны}\\
\end{center}
\begin{flushleft}
 \textbf{Работу выполнили:} \\
Аверьянов Тимофей, Козлова Елизавета, Корякин Алексей, Маслова  Анна.
 \end{flushleft}
 
Агент \textcolor[rgb]{0.82,0.01,0.11}{А} - компания;

Агента \textcolor[rgb]{0.29,0.56,0.89}{B} - государство;

Действия игрока \textcolor[rgb]{0.82,0.01,0.11}{А}\textcolor[rgb]{0,0,0}{:}
\begin{itemize}
\item \textcolor[rgb]{0.82,0.01,0.11}{A1} - вывод капитала в офшоры через осуществление фиктивных сделок;
\item \textcolor[rgb]{0.82,0.01,0.11}{А2} - манипуляции при внешнеторговых операциях;
\item \textcolor[rgb]{0.82,0.01,0.11}{А3} - выдача дочерней компанией заемных средств материнской компании, которые облагаются по пониженным налоговым ставкам офшоров;
\item \textcolor[rgb]{0.82,0.01,0.11}{А4} - регистрация дочерних компаний в оффшорных финансовых центрах;
\item \textcolor[rgb]{0.82,0.01,0.11}{А5} - перевод средств в российскую оффшорную зону.
\end{itemize}

Действия игрока \textcolor[rgb]{0.29,0.56,0.89}{B}\textcolor[rgb]{0,0,0}{:}
\begin{itemize}
\item \textcolor[rgb]{0.29,0.56,0.89}{B1} - ужесточение внутреннего законодательства, штрафы, политическое давление;
\item \textcolor[rgb]{0.29,0.56,0.89}{B2} - заключение соглашений с офшорами об обмене налоговой информацией;
\item \textcolor[rgb]{0.29,0.56,0.89}{B3} - создание офшорной зоны в России;
\item \textcolor[rgb]{0.29,0.56,0.89}{B4} - отмена соглашений об избежании двойного налогообложения;
\item \textcolor[rgb]{0.29,0.56,0.89}{B5} - создание специальных списков конечных бенефициаров оффшорных компаний;
\item \textcolor[rgb]{0.29,0.56,0.89}{B6} - отказ от борьбы с офшорами.
\end{itemize}

\begin{table}[H]
        \centering
        
\begin{tabular}{|p{0.1\textwidth}|p{0.1\textwidth}|p{0.1\textwidth}|p{0.1\textwidth}|p{0.1\textwidth}|p{0.1\textwidth}|p{0.1\textwidth}|}
\hline 
 \textcolor[rgb]{0.82,0.01,0.11}{A}\textbackslash \textcolor[rgb]{0.29,0.56,0.89}{B} & \textcolor[rgb]{0.29,0.56,0.89}{B1} & \textcolor[rgb]{0.29,0.56,0.89}{B2} & \textcolor[rgb]{0.29,0.56,0.89}{B3} & \textcolor[rgb]{0.29,0.56,0.89}{B4} & \textcolor[rgb]{0.29,0.56,0.89}{B5} & \textcolor[rgb]{0.29,0.56,0.89}{B6} \\
\hline 
 \textcolor[rgb]{0.82,0.01,0.11}{A1} & -1; +3 & -1; +2 & +1; +4 & -3; +4 & -3; +2 & +3; 0 \\
\hline 
 \textcolor[rgb]{0.82,0.01,0.11}{A2} & +1; +2 & +2; +3 & +1; +1 & 0; +3 & 0; 0 & +4; 0 \\
\hline 
 \textcolor[rgb]{0.82,0.01,0.11}{A3} & -3; +2 & -2; +2 & -2; +1 & 0; 0 & -3; +2 & +4; -2 \\
\hline 
 \textcolor[rgb]{0.82,0.01,0.11}{A4} & -3; +2 & -3; +4 & -1; +2 & -1; +1 & -3; +2 & +4; -2 \\
\hline 
 \textcolor[rgb]{0.82,0.01,0.11}{A5} & +3; +3 & +3; +1 & \cellcolor{yellow!50}{+4; +4} & +3; +1 & +2; +3 & -2; 0 \\
 \hline
\end{tabular}
        
        \end{table}
        
Точка равновесия по Нэшу: [+4; +4]
    \end{document}
