\documentclass[12pt,a4paper]{article}
\usepackage[14pt]{extsizes}
\usepackage[utf8]{inputenc}
\usepackage{amsmath}
\usepackage{amsfonts}
\usepackage{amssymb}
\usepackage{cmap}
% for fonts
    \usepackage[T2A, T1]{fontenc}
    \usepackage[english, russian]{babel}
    \usepackage{fontspec}
    \defaultfontfeatures{Ligatures=TeX,Renderer=Basic}
    \setmainfont[Ligatures={TeX, Historic}]{Times New Roman}
    \setsansfont{Times New Roman}
    \setmonofont{Courier New}
% mathcha
\usepackage{tikz}
\usepackage{mathdots}
\usepackage{yhmath}
\usepackage{cancel}
\usepackage{color}
\usepackage{siunitx}
\usepackage{array}
\usepackage{multirow}
\usepackage{amssymb}
\usepackage{gensymb}
\usepackage{tabularx}
\usepackage{booktabs}
\usetikzlibrary{fadings}
% mathcha
\usepackage{pgfplots} % plot
\usepackage{float} % for H at figure
\usepackage{cases}
\pgfplotsset{compat=1.15}
\usepackage{graphicx}
\usepackage[left=2cm,right=2cm,top=2cm,bottom=2cm]{geometry}
\author{Аверьянов Тимофей, Корякин Алексей}
\begin{document}
\begin{center}
\section*{Семинар №7}

\textbf{Производственная функция фирмы и её примеры: Кобба-Дуглоса, линейная и Леонтьева}

\textbf{План}
\end{center}

\begin{enumerate}
\item Понятие производственной функции фирмы, её основные свойства и три примера;
\item Основные характерстики производственной функции: предельные продукты факторов производства, эластичность выпуска по факторам производства, средние продукты факторов, изокванты производственной функции;
\item $\displaystyle \boxed{\text{ДЗ}}$
\end{enumerate}

Приступаем к моделированию поведения фирмы при производстве благ (товар или услуга). Обозначим символом $\displaystyle y$ \textit{кол-во продукции (блага)}, которое производит фирма на заданном отрезке времени; в процессе производства этого блага фирма использует \textit{факторы производства} и уровни этих факторов мы будем обозначать сиволоми: $\displaystyle x_{1} ,\ \dotsc ,\ x_{n}$. Пусть $\displaystyle x_{1} -$основной капитал средство производства, второй фаткор $\displaystyle x_{2} \ -\ $это кол-во живого труда, $\displaystyle x_{3} -$запасы, полуфабрикаты, $\displaystyle x_{n} \ -$ финансовый капитал.

В процессе выбора фирма использует технология $\displaystyle f$ при помощи которой факторы производства трансформируются в уровень продукции $\displaystyle y$. Вот лаконичная запись такой трансформации:
\begin{equation*}
y,\ x_{1} ,\ \dotsc ,\ x_{n} ;\ f
\eqno(1)
\end{equation*}
\begin{equation*}
y\ =\ f\ ( x_{1} ,\ \dotsc ,\ x_{n})
\eqno(2)
\end{equation*}
Математическое выражение (2) называется \textit{производственной функцией фирмы}.

Отметим основые \textbf{свойства} производственной функции:
\begin{enumerate}
\item При нулевых факторах производства, то и выпуск 0. $\displaystyle f( 0,0,\dotsc ,0) =0$;
\item Производственная функция возрастает по каждому фактору производства $\displaystyle f\uparrow x_{i}$, т.е. $\displaystyle f\uparrow x_{i} \Leftrightarrow M_{y}( x_{i})$. $\displaystyle M_{y}( x_{i}) -$предельное значение выпуска по $\displaystyle i$-ому фактору, т.е. это дополнительный выпуск продукции в ответ на дополнительную еденицу $\displaystyle i$-ого фактора.
\item С ростом уровня $\displaystyle x_{i} \uparrow $ фактора его предельный выпуск убывает $\displaystyle M_{y}( x_{i}) \downarrow $. Каждая дополнительная еденица фактора менее полезна, чем предыдущая дополнительная еденица.
\end{enumerate}

Приведем три примера производсвенный функций, удовлетворяющих в той или иной мере своствам производственной функции.

Пример 1. $\displaystyle y\ =\ f\left( x_{1} \ \left( =\text{основной капитал}\right) ,\ x_{2}\left( =\text{живой труд}\right)\right)$.
\begin{equation*}
\boxed{ \begin{array}{{>{\displaystyle}l}}
y\ =\ a_{0} \ \cdot x_{1} \cdot x_{2}\\
a_{0} \  >0,\ 0\ < a_{1} \ < 1;0< a_{2} < 1
\end{array}}
\eqno(4)
\end{equation*}
Производсвенная фукций (4) называется \textit{производственной функцией Кобба-Дуглоса.}

Пример 2. Линейная проиводственная функция:
\begin{equation*}
  \begin{gathered}
    y\ =\ a_{1} x_{1} \ +\ a_{2} x_{2}\\
    a_{1}  >0;\ a_{2}  >0
  \end{gathered}
  \eqno(5)
\end{equation*}
Пример 3. Производсвенная функция Леонтьева.
\begin{equation*}
  \begin{gathered}
    y=\ a_{0}\min( x_{1} ,\ x_{2})\\
    a_{0}  >0
  \end{gathered}
  \eqno(6)
\end{equation*}
$\displaystyle \boxed{\text{ДЗ №1}}$ Проверить справедливость свойств производственной функции для (4), (5), (6).

\textbf{Основные характерстики производсвенной функции Коббла-Дугласа.}
\begin{equation*}
y\ =0.45x^{0.5}_{1}\left(\text{млрд. долл.}\right) x^{0.1}_{2}\left(\text{тыс. человек}\right) \eqno(4')
\end{equation*}
1. Предельный выпуск фирмы по правилу производсва: $\displaystyle M_{y}( x_{i}) \ \approx \frac{\partial f}{\partial x_{i}}( 7)$. Также называют \textit{предельным продуктом фактора }$\displaystyle x_{i}$\textit{.}

\textbf{Решим задачу.} Определить уравнение предельного продукта $\displaystyle M_{y}( x_{i})$ по первому фактору и рассчитать значние этого предельного продукта $\displaystyle x_{1} =6,\ x_{2} =17$:
\begin{equation*}
M_{y}( x_{i}) \approx \frac{\partial f}{\partial x_{1}} \ =0.45\cdot 0.5\cdot x^{0.5-1}_{1} x^{0.1}_{2} =0.45\cdot 0.5\ x^{-0.5}_{1} x^{0.1}_{2} =0.45\cdot 0.5\ 6^{-0.5} 17^{0.1} =0.12
\eqno(8)
\end{equation*}
Величина $\displaystyle M_{y}( x_{i}) \ =0.12\ -$это значение дополнительного выпуска продукции в ответ использование в процессе произвоства дополнительной еденицы совокупного капитала.

$\displaystyle \boxed{\text{ДЗ № 2}}$ Получить уравнение предельного продукта второго фактора и дать интерпритацию.

\textbf{При помощи уранения (8) в }$\displaystyle \boxed{\text{ДЗ № 1}}$\textbf{ \ можно проверить справедливость второго свойства проиводсвенной функции.}

\textbf{Средний продукт фактора производства}.

2. Средним продуктом фактора производства $\displaystyle A_{y}( x_{i})$ экономисты называют дробь $\displaystyle \frac{y}{x_{i}}$.
\begin{equation*}
A_{y}( x_{i}) \ =\frac{y}{x_{i}}
\eqno(9)
\end{equation*}
Средний продукт фактора производсва - это кол-во выпуска продукции приходящаяся на одну единицу данного фактора.

\textbf{Задача №2. }Для функции Кобба-Дугласа вычислить значение предельного продукта первого фактора и посчитать его значение применитльно уравнения (4').
\begin{equation*}
A_{y}( x_{1}) \ =\frac{y}{x_{1}} =\frac{0.45\cdot 6^{0.5} \cdot 17^{0.1}}{6} \ =0.24
\eqno(11)
\end{equation*}
У данной фирмы на одну единицу совокупного капитала приходится в среднем 0.24 млрд. продукции.

$\displaystyle \boxed{\text{ДЗ № 3}}$ Для функции Кобба-Дугласа вычислить значение предельного продукта \textbf{второго} фактора и посчитать его значение применительно уравнения (4').

3. \textbf{Эластичность выпуска по факторам производства }расчитаывается по правилу (смотри занятие №2):
\begin{equation*}
E_{y}( x_{i}) =M_{y}( x_{i}) :A_{y}( x_{i}) \
\end{equation*}
\textbf{Задача №3. }Расчитать значение эластичности в рамках формулы (4') по формуле (11):
\begin{equation*}
E_{y}( x_{1}) =0.12:0.24=0.5
\end{equation*}
Эластичность выпуска функции Коббла-Дугласа равна показателю степени $\displaystyle a_{1}$. Следовательно коэффициент $\displaystyle a_{1}$ в уравнении (4') - это эластичность функции выпуска.

$\displaystyle \boxed{\text{ДЗ № 4}}$ Определить эластичность выпуска по второму фактору.

4. \textbf{Изокванты }заданного уровня $\displaystyle y_{0}$ экономисты называют \textit{линию уровня функции, }т.е. множество различных комбинаций факторов производства при которых уровень выпуска продукции остаётся неизменным и равным заданной велечине $\displaystyle y_{0}$. Изокванту удобно изучать разрешив уравнение (13) относительно переменной например $\displaystyle x_{1}$, т.е. привратив переменную $\displaystyle x_{1}$ в функцию переменной $\displaystyle x_{2}$ зависящей, как от параметра.

\textbf{Задача №4. }Получить уравнение изокванты и построить график этой изокванты для функции (4') принимая значение $\displaystyle y_{0} =2$.

\textbf{Решение.} Аналогом изокванты является функция безразличия и мы можем воспользоваться домашней задачей № 3.
\begin{equation*}
x_{1} \ =\ \left(\frac{y}{a_{0} \ \cdot x^{a_{2}}_{2}}\right)^{\frac{1}{a_{1}}} =\ \left(\frac{y_{0}}{a_{0}}\right)^{\frac{1}{a_{1}}} \cdot x^{-\frac{a_{2}}{a_{1}}}_{2} \ ( x_{2}  >0)
\end{equation*}
\end{document}
