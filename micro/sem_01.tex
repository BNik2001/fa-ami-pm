\documentclass[12pt,a4paper]{article}
\usepackage[14pt]{extsizes} 
\usepackage[utf8]{inputenc}
\usepackage{amsmath}
\usepackage{amsfonts}
\usepackage{amssymb}
% for fonts
    \usepackage[T2A, T1]{fontenc}
    \usepackage[english, russian]{babel}
    \usepackage{fontspec}
    \defaultfontfeatures{Ligatures=TeX,Renderer=Basic}
    \setmainfont[Ligatures={TeX, Historic}]{Times New Roman}
    \setsansfont{Times New Roman}
    \setmonofont{Courier New}
\usepackage{pgfplots} % plot
\pgfplotsset{compat=1.15}
\usepackage{graphicx}
\usepackage[left=2cm,right=2cm,top=2cm,bottom=2cm]{geometry}
\author{Аверьянов Т.С.}
\begin{document}

\begin{center}
\section*{Метод математического моделирования изучения экономики.}
\end{center}

\begin{center}
\large{\textbf{План}}
\end{center}
\begin{enumerate}
\item Известные и искомые характеристики изучаемого объекта, запись взаимосвязей этих характеристик математическим языком. Спецификация (подробное описание математической модели) модели Баумоля-Тобина спроса на наличные деньги (модель управления наличностью, модель оптимального остатка денежных средств на счёте);
\item Трансформация модели Баумоля-Тобина к приведённой форме методом Лагранжа;
\item Домашнее задание;
\end{enumerate}

Изучение экономики (и реально мира вобще) базируется на записи мат. языком взаимосвязей известных характеристик изучаемого объекта (экзогенных переменных) и искомых характеристик (эндогенных переменных). Такая запись именуется записью \textbf{экономической моделью} и в этой модели искомые и известные характеристики связаны между собой воедино. В процессе записи математическим языком возникает структурная форма модели и если во взаимосвязях содержится некоторое требование оптимальности у искомым значениям эндогенных переменных, то такая модель называется оптимизационной.

\begin{equation}
 \begin{cases}
   P(\vec{y}, \vec{x}) \rightarrow ext (\min |  \max) \\
   \vec{y} \in Y_{\vec{x}}
 \end{cases}
\end{equation}

В верхней строчке записано требование оптимальности искомых значений к эндогенным переменным. \\
В экономики (везде) требование минимальных издержек или требование максимального дохода. \\
Во второй строчке лаконично записано условие допустимости значения эндогенных переменных, которое это условие содержит также во взаимосвязях. Символом $Y$ мы обозначили множество допустимых значений $\vec{y}$ и это множество в общем случае зависит от экзогенных переменных $\vec{x}$. В математике такие задачи называются \textit{задачами математического программирования}. Добавим, что слева от стрелки находится функция экзогенных и эндогенных переменных, которая в математике называется \textit{целевой} и в экономике значение этой функции всегда имеют смысл, либо издержек, либо дохода. \\
\textbf{Задача Баумоля-Тобина.} Изучаемым объектом является опреционная деятельность (по производству сметаны), которая требует в наличных денег.
\begin{enumerate}
\item $M$ - требуемый уровень денежных средств в течение года ($M$ = \$52 млн.) Для обеспечения денежными средствами фирма в начале года открывает в банке расчетный счёт и размещает на этом счёте некоторое кол-во денег $m$. Как правило эти деньги фирма берёт в кредит или же получает в итоге продажи своих ценных бумаг. При такой продаже фирма имеет издержки на известном уровне $c$ малое.
\item $c$ - величина издержек. ($c$ = \$0.05 млн.) Деньги $m$ находящиеся на расчётном счёте не приносят ей доход, а между тем, если эти деньги фирма разместила на депозите (инвестировала в депозит), то эти деньги приносили бы доход $r$ малое и называется у экономистов нормой альтернативных затрат.
\item $r$ - норма альтернативных затрат ($r$ = 0.07 = 7\%). Деньги размещённые на счёте фирмы не приносят доход и этот доход носит название альтернативных затрат, альтернативные затраты всегда экономисты включают в общие затраты фирмы. Таким образом исходными данными являются:
\end{enumerate}
\begin{itemize}
\item $M$ - требуемый уровень денежных средств в течение года ($M$ = \$52 млн.)
\item $c$ - величина издержек. ($c$ = \$0.05 млн.)
\item $r$ - норма альтернативных затрат ($r$ = 0.07 = 7\%)
\end{itemize}
Искомыми величинами:
\begin{enumerate}
\item Величина остатка денежных средств на счёте в момент его пополнения (m)
\item Кол-во пополнений (на рис ниже $n$ = 4)
\end{enumerate}

\begin{tikzpicture}
\begin{axis}[
        xlabel={$t$},
        ylabel={$v$},
        axis lines=center,
        symbolic x coords={0,3,6,9,12,15},
        xmin={0},
        xmax={15},
        xtickmax={12},
        xtick distance=1,
        symbolic y coords={0,$m$,$M$},
        ymin={0},
        ymax={$M$},
        ytickmax={$m$},
        ytick distance=1]
\addplot [color=red,mark=*] coordinates {(0,{$m$}) (3,0)};
\addplot [color=red,mark=*] coordinates {(3,$m$) (6,0)};
\addplot [color=red,mark=*] coordinates {(6,$m$) (9,0)};
\addplot [color=red,mark=*] coordinates {(9,$m$) (12,0)};
\draw [dashed,help lines] (axis cs:3,0) -- (axis cs:3,$m$);
\draw [dashed,help lines] (axis cs:6,0) -- (axis cs:6,$m$);
\draw [dashed,help lines] (axis cs:9,0) -- (axis cs:9,$m$);
\end{axis}
\end{tikzpicture}

В начале года остаток m, по мере расчёта остаток снижается до 0 и затем пополняется. m и n - эндогенные переменные. Взаимосвязи отражены словесно:
\begin{enumerate}
\item Общие затраты фирмы ($\phi$) должны быть минимальными.
\item Величины m и n должны быть такими чтобы они удовлетворяли требуемему уровню М.
\end{enumerate}

Начнём с записи общих затрат $\phi$ помня, что эти слагаемые состоят из двух частей:
$\phi_1$ и это слагаемое состоит из общей величины издержек $\phi_1 = c \times n$, второе слагаемое это упущенный доход (альтернативные издержки) ($\phi_2$) \\
$\phi = \phi_1 + \phi_2$\\
$\phi_1 = c \cdot n,$\\
$\phi_2 = \frac{m}{2} \cdot r$\\

Оптимизированная модель Баумоля в структурной форме:

\begin{equation}
	\begin{cases}
	\phi = c \cdot n + \displaystyle{\frac{r}{2}} \cdot m \rightarrow \min \\
	n \cdot m = M, \\
	n \geq 0, m \geq 0.
	\end{cases}
\end{equation}

С точки зрения математики оптимизационная модель Баумоля-Тобеля отпносится к классическим задачам математического программирования на условный экстремум. Решить такую задачу означает трансформировать модель к приведённой форме. \\

Метод Лагранжа состоит из 3 шагов:
\begin{itemize}
\item Составляется функция Лагранжа на условный экстремум:
\begin{equation}
L = c \cdot n + \frac{r}{2} \cdot m + l \cdot (M - n \cdot m)
\end{equation} 
В функции символом $l$ обозначен множитель Лагранжа.
\item Для функции Лагранжа все производные должны быть равны нулю. 
\begin{equation}
\begin{cases}
\displaystyle{\frac{\partial{L}}{\partial{n}}} = c - l \cdot m = 0, \\[10pt]
\displaystyle{\frac{\partial{L}}{\partial{m}}} = \frac{r}{2} - l \cdot n = 0,\\[10pt]
\displaystyle{\frac{\partial{L}}{\partial{l}}} = M - n \cdot m = 0
\end{cases}
\end{equation}
\item Составленная система решается численно (на практике как правило), либо аналитически.
\end{itemize}

Приведённая форма модели Баумоля:\\
Формулы Уилсона
\begin{equation}
\begin{cases}
m = \displaystyle{\frac{c}{l}}, \\[10pt]
n = \displaystyle{\frac{r}{2 \cdot l}}, \\[10pt]
M - \displaystyle{\frac{r}{2 \cdot l}} \cdot \displaystyle{\frac{c}{l}} = 0 \Rightarrow l^2 = \displaystyle{\frac{r \cdot c}{2 \cdot M}} \Rightarrow l = \sqrt{\displaystyle{\frac{r \cdot c}{2 \cdot M}}}.
\end{cases}
\end{equation}

Приведённая форма модели (формулы Уилсона):
\begin{equation}
m^*=\sqrt{\displaystyle{\frac{2 \cdot c \cdot M}{r}}}, n^*=\sqrt{\displaystyle{\frac{r \cdot M}{2 \cdot c}}}
\end{equation}

\framebox[1.1\width]{Д/з.} C упомянутыми значениями экзогенных переменных расчитать эндогенные значения $m^*$ и $n^*$. Подставить правые части формул Уилсона в уравнение формул издержек и получить значение, как явную функцию экзогенных переменных. Отдельно рассчитать издержки и упущенный доход ($\phi_1$ и $\phi_2$). Рассчитать по формуле значение множителей Лагранжа. 




\end{document}