\documentclass[12pt,a4paper]{article}
\usepackage[14pt]{extsizes}
\usepackage[utf8]{inputenc}
\usepackage{amsmath}
\usepackage{amsfonts}
\usepackage{amssymb}
\usepackage{cmap}
% for fonts
    \usepackage[T2A, T1]{fontenc}
    \usepackage[english, russian]{babel}
    \usepackage{fontspec}
    \defaultfontfeatures{Ligatures=TeX,Renderer=Basic}
    \setmainfont[Ligatures={TeX, Historic}]{Times New Roman}
    \setsansfont{Times New Roman}
    \setmonofont{Courier New}
%plot
\usepackage{pgf,tikz,pgfplots}
\pgfplotsset{compat=1.15}
\usepackage{mathrsfs}
\usetikzlibrary{arrows}
\pagestyle{empty}
%plot
\usepackage{float}% for \begin{figure}[H]
\usepackage{cases}
\usepackage{graphicx}
\usepackage[left=2cm,right=2cm,top=2cm,bottom=2cm]{geometry}
\author{Аверьянов Тимофей, Корякин Алексей}
\begin{document}
\begin{center}
\section*{Семинар №4 \\
Модель Маршала-Лаграса. Модель поведения потребителя на рынке благ}
\end{center}
\begin{center}
\textbf{План}
\end{center}
\begin{enumerate}
\item Структурная форма Маршала-Лаграса
\item Функция спроса потребителя и её свойства. Функция косвенной полезности потребителя.
\end{enumerate}
\begin{center}
\textbf{Подзаголовок}
\end{center}

На прошлом занятии обсудили понятие функции полезности и отметили её свойства. Это понятие мы привлечем в процессе обсуждения модели поведения потребителя. Экономическая суть это модели следующая: потребитель приобретает такой набор благ, который, с одной стороны, ему максимально полезнен, а с другой стороны – по карману
\begin{equation}
\vec(x)^*=(x_1, ..., x_n)
\begin{cases}
u=u(x_1, ..., x_n) -> max; \\
\sum_{i=1}^n p_i x_i; \\
x_1 > 0, ...., x_n > 0;
\end{cases}
\end{equation}

Экзогенными перменными являются бюджет потребителя и цены благ.
Эндогенными – количество благ.
Функция полезности задается экзогенно.
Модель 1 с позиции математики является примером задачи математического программирования на условный экстремум; с позиции метода математического моделирования – это оптимизационная модель в структурной форме (см. лекцию и семинар №1).

К приведенной форме модель 1 трансформируется методом Лагранжа.
\begin{enumerate}
  \item Составляется функция Лагранжа. $L=u(x_1, ..., x_n)+l(M-\sum_{i}^n p_i x_i)$
  \item Cоставляется условие экстремума $\frac{\partial L}{\partial x_i}=0 $. \\
  Условия представляют собой систему $n+1$ уравнения с $n+1$ неизвестными.
  \item Система 3 решается либо аналитчески, либо численно.
\end{enumerate}
Итогом решения является искомый набор $\vec(x)=\vec(x)^{M-B}(M, p_1, ..., p_n)$
\textbf{Пример.} Текст
\begin{itemize}
\item Пункт
\item Пункт
\end{itemize}

\framebox[1.1\width]{Д/з}. Задание

\end{document}
