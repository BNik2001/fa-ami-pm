\documentclass[12pt,a4paper]{article}
\usepackage[14pt]{extsizes} 
\usepackage[utf8]{inputenc}
\usepackage{amsmath}
\usepackage{amsfonts}
\usepackage{amssymb}
\usepackage{cmap}
% for fonts
    \usepackage[T2A, T1]{fontenc}
    \usepackage[english, russian]{babel}
    \usepackage{fontspec}
    \defaultfontfeatures{Ligatures=TeX,Renderer=Basic}
    \setmainfont[Ligatures={TeX, Historic}]{Times New Roman}
    \setsansfont{Times New Roman}
    \setmonofont{Courier New}
%plot
\usepackage{amsmath}
\usepackage{tikz}
\usepackage{mathdots}
\usepackage{yhmath}
\usepackage{cancel}
\usepackage{color}
\usepackage{siunitx}
\usepackage{array}
\usepackage{multirow}
\usepackage{amssymb}
\usepackage{gensymb}
\usepackage{tabularx}
\usepackage{booktabs}
\usetikzlibrary{fadings}
%plot
\usepackage{float}% for \begin{figure}[H]
\usepackage{cases}
\usepackage{graphicx}
\usepackage[left=2cm,right=2cm,top=2cm,bottom=2cm]{geometry}
\author{GH-TIMe}
\begin{document}

\begin{center}
\section*{Семинар №2: Предельные величины в экономике и значение эластичности}
\end{center}

\begin{center}
\textbf{План}
\end{center}
\begin{enumerate}
\item Расчёт предельных издержек фирмы на поддержание расчётного счёта (при помощи модели Баумоля-Тобина);
\item Расчёт эластичности издержек фирмы по поддержанию расчётного счёта;
\item Обсуждение ДЗ;
\end{enumerate}

На прошлом занятии познакомились с методом математического моделирования изучения экономики в процессе составления и расчётов модели Баумоля управления наличностью фирмы:

\begin{equation}
	\begin{cases}
	\phi = c \cdot n + \displaystyle{\frac{r}{2}} \cdot m \rightarrow \min \\
	n \cdot m = M, \\
	n \geq 0, m \geq 0.
	\end{cases}
\end{equation}

Подчеркнём, что это структурная форма модели; Методом Лагранджа эту форму мы трансформировали к приведённой форме:

\begin{equation}
m^*=\sqrt{\displaystyle{\frac{2 \cdot c \cdot M}{r}}}, n^*=\sqrt{\displaystyle{\frac{r \cdot M}{2 \cdot c}}}
\end{equation}

Формулу (2) можно использовать для проверки размерности. В домашнем задании получена приведённая форма. 

\begin{equation}
\phi = \sqrt{2crM}
\end{equation}

Подчеркнём, что каждая эндогенная переменная выражена через экзогенные 
\begin{equation}
(M, C, r)
\end{equation}


\textit{Предельными величинами} в экономике принято называть изменения эндогенных переменных возникающие в ответ на единичные изменения экзогенных переменных. 

Познакомимся с этим понятием в процессе решения \textbf{задачи}:\\
Пусть $M = \$52$ млн.\\
$c = \$ 0,05$ млн. \\
$r = 0,07 = 7\%$\\
Требуется определить: \\
\begin{equation}
\triangle \phi = \phi(M+1(\triangle M), c, r) - \phi(M,c,r)
\end{equation}
Величина $\triangle \phi$ и есть предельный величинина.\\
\textbf{Решение:}Прежде всего обратим внимание, что величина $\triangle \phi$ -- это частное приращение функции, возникающее в ответ на изменение аргумента $M$ на величину $\triangle M = 1$. Вычислим его при заданных значениях экзогенных переменных.

$$\triangle \phi = 0,6090977 - 0,6033241 = 0,0057736 \approx 0,0058$$

Имея ввиду смысл величины $\phi$ мы можем сказать, что величина $\triangle \phi$ вычисленная по правилу (5) имеет смысл \textit{цены денежных ресурсов} фирмы, более точно это \textit{предельная цена ресурса}. Мы обозначим эту величину $\phi = M_{\phi}(M) = 0,0058$. 

Правила расчёта предельных велечин в экономике. Вернёмся к выражению (5) и воспользуемся понятием дифференциала функции, как главной части приращения 
\begin{equation}
\triangle \phi = \frac{\partial \phi}{\partial M} \cdot 1(\triangle M)
\end{equation}

Именно при помощи дифференциала (частного дифференциала) все прикладники вычисляют приращение функции при помощи дифференциала. Рис.1. иллюстрирует формулу (6). 

\tikzset{every picture/.style={line width=0.75pt}} %set default line width to 0.75pt        

\begin{figure}[H]
\begin{center}
\begin{tikzpicture}[x=0.75pt,y=0.75pt,yscale=-1,xscale=1]
%uncomment if require: \path (0,300); %set diagram left start at 0, and has height of 300

%Shape: Axis 2D [id:dp5980054185251586] 
\draw  (18,160.28) -- (293.3,160.28)(45.53,17) -- (45.53,176.2) (286.3,155.28) -- (293.3,160.28) -- (286.3,165.28) (40.53,24) -- (45.53,17) -- (50.53,24)  ;
%Curve Lines [id:da4502339839588414] 
\draw    (52.63,160.28) .. controls (119.3,102.2) and (199.3,62.2) .. (275.3,62.2) ;


%Straight Lines [id:da9290270254581325] 
\draw  [dash pattern={on 4.5pt off 4.5pt}]  (30.3,160.2) -- (288.3,17.2) ;


%Straight Lines [id:da5200705054945007] 
\draw  [dash pattern={on 4.5pt off 4.5pt}]  (121.3,109.2) -- (121.3,161.2) ;


%Straight Lines [id:da26961713863509607] 
\draw  [dash pattern={on 4.5pt off 4.5pt}]  (191.3,77.2) -- (191.3,160.2) ;


%Straight Lines [id:da2330222678621816] 
\draw    (121.3,109.2) -- (191.3,109.2) ;



% Text Node
\draw (156,121) node  [align=left] {$\displaystyle \vartriangle M\ $};
% Text Node
\draw (165,95) node [scale=0.8] [align=left] {$\displaystyle \vartriangle \phi $};
% Text Node
\draw (118,173) node  [align=left] {M};
% Text Node
\draw (187,174) node  [align=left] {M+1};
% Text Node
\draw (184,95) node [scale=2.074,rotate=-180.25] [align=left] {\}};
% Text Node
\draw (199,86) node [scale=2.488,rotate=-0.11,xslant=0.01] [align=left] {\}};
% Text Node
\draw (274,148) node  [align=left] {M};
% Text Node
\draw (66,22) node  [align=left] {$\displaystyle \phi $};
\end{tikzpicture}
\end{center}
\caption{Иллюстрация формулы (6)}
\end{figure}

\textbf{Задача 2}. Вычислить издержки по правилу (6) и сравнить с точным.

\textbf{Решение:}\\
$$\frac{\partial \phi}{\partial M} = \left(\frac{1}{2}\sqrt{2 c r M}\right) \cdot 2cr = \sqrt{\frac{cr}{2M}} = \sqrt{\frac{0,05 \cdot 0.07}{2 \cdot 52}} = 0,0058012$$

Сопоставляя расчёты по формулам (5) и (6) мы убеждаемся в достаточной точности формулы (6), которая использунтся во всех приложениях. Предельные значения эндогенных переменных принято вычислять, как частные производные эндогенных переменных по экзогенным.

Добавим к сказанному, что при помощи дифференциала, также удобно вычислять изменения по поддержанию счёта в ответ на заданные изменения любой экзогенной переменной , например дополнительные издержки, которые возникают в ответ на одну транзакцию, удобно посчитать по правилу 
\begin{equation}
\triangle \phi = \frac{\partial \phi}{\partial c} \cdot \triangle c
\end{equation}

\textbf{Задача 3}. На прошлом занятии трансформируя модель (1) к приведённой форме (2) мы определили значениме множителя Лагранжа $l$ и вычислили в ДЗ на прошлом занятии.

\begin{equation}
\frac{\partial \phi }{\partial M} \ =\ l\ =\ \sqrt{\frac{cr}{2M}}
\end{equation}

Видим, что выражению (8) предельные издержки по $M$ это ни что иное, как множитель Лагранжа. Следовательно, множитель Лагранжа имеет экономический смысл предельной цены рессурса $M$. Множитель Лагранжа имеет смысл предельной цены денежных средств.

\begin{center}
 \textbf{Эластичность в экономике}.
 \end{center} 

Значения эластичности -- это велечины, которые связывают \underline{относительные} изменения эндогенных переменных в ответ на заданные относительные изменения экзогенных переменных. С понятием эластичности познакомимся в итоге решения следующей \textbf{задачи}:

Пусть фирме потребовались дополнительные денежные ресурсы в размере 3\% от принятого ранее уровня денежных ресурсов. Спрашивается на сколько процентов в ответ возрастёт уровень оптимальных затрат фирмы по поддежанию счёта?

Вернёмся к определению эластичности и запишем это определение математическим языком 
\begin{equation}
\frac{\triangle \phi}{\phi} = E_\phi(M) \frac{\triangle M}{M}
\end{equation}
это выражение мы можем переписать так:
\begin{equation}
E_\phi(M) = \frac{\triangle \phi}{\phi} : \frac{\triangle M}{M} = \frac{\triangle \phi}{\triangle M} \frac{M}{\phi} = \frac{\triangle \phi}{\triangle M} : \frac{\phi}{M}
\end{equation}

Последний член $\displaystyle{\frac{\phi}{M}}$ имеет \textit{смысл средней цены денежных средств}, т.е. это тот уровень издержек, который приходится на одну единицу требуемых денежных средств. Мы обозначим эту велечину
\begin{equation}
\frac{\phi}{M} = A_{\phi} (M)
\end{equation}

Формулу (10) легко запомнить, а именно эластичность -- это отношение предельных издержек к средним.

\textbf{Решение:} Вернёмся к выражению (9) множитель, которой равен $\displaystyle{\frac{\triangle M}{M} = 3\%}$, выразим эластичность по формуле (10):

\begin{equation*}
E_{\phi }( M) \ =\ \frac{\vartriangle \phi }{\phi } :0,03\ =\ \frac{0,0058}{0,6} \ :0,03\ =\ 0,5
\end{equation*}

Значение $E_{\phi}(M) = 0,5$ имеет следующий смысл: если велечину $M$ увеличить на 1\%, то оптим. изд. фирмы возрастут на полпроцента.

Значение $E_{\phi}(M)$ имеет следующий смысл: относительное изменение $\phi$ в ответ на изменение велечины ($M$) на 1\%. 


\framebox[1.1\width]{Д/з} Вычислить предельное значение эндогенных переменных $m$ и $n$ по экозогенной переменной $M$ и дать трактовку $m$ и $n$. 

Пусть транзакционные издержки возрастают на 2\% во сколько в ответ в относительной мере (\%) увеличится велечина $m$.

\end{document}
