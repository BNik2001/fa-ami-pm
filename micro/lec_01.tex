\documentclass[12pt,a4paper]{article}
\usepackage[14pt]{extsizes} 
\usepackage[utf8]{inputenc}
\usepackage{amsmath}
\usepackage{amsfonts}
\usepackage{amssymb}
\usepackage{cmap}
% for fonts
    \usepackage[T2A, T1]{fontenc}
    \usepackage[english, russian]{babel}
    \usepackage{fontspec}
    \defaultfontfeatures{Ligatures=TeX,Renderer=Basic}
    \setmainfont[Ligatures={TeX, Historic}]{Times New Roman}
    \setsansfont{Times New Roman}
    \setmonofont{Courier New}
%plot
\usepackage{pgf,tikz,pgfplots}
\pgfplotsset{compat=1.15}
\usepackage{mathrsfs}
\usetikzlibrary{arrows}
\pagestyle{empty}
%plot
\usepackage{float}% for \begin{figure}[H]
\usepackage{cases}
\usepackage{graphicx}
\usepackage[left=2cm,right=2cm,top=2cm,bottom=2cm]{geometry}
\author{GH-TIMe}
\begin{document}

\begin{center}
\section*{Лекция №1: Метод математического моделирования в экономике.}
\end{center}

\textbf{Список литературы:}
\begin{itemize}
\item Экономико-математическре моделирование Дрогобыцкого И.Н.
\item Интрилигатор М. Математические методы оптимимзации и экономическая теория.
\item Нуреев Р.М. Курс микроэкономики: учебник.
\end{itemize}

\begin{center}
\textbf{План}
\end{center}

\begin{enumerate}
\item Экономика как объект изучения и как наука (Нуреев);
\item Метод математического моделирования экономики (Модель, типы переменных в моделе, два класса моделей, две формы модели);
\item Предельные величны и эластичность в экономике;
\end{enumerate}

\begin{center}
\textbf{Экономика как объект изучения и как наука}
\end{center}

Экономика как объект изучения предстваляет собой совокупность или множество институтов, деятельность которых направлена на деятельность удовлетворения потребностей населения в ситуации ограниченных ресурсов. Основными объектами микроэкономики являются:
\begin{enumerate}
\item Фирмы, производящие блага (товары или услуги) и продающие эти блага на рынке;
\item Домашние хозяйства являющиеся потребителями благ и в нашем курсе мы будем изучать методом математического моделирования поведение потребителей благ и фирм при их взаимодействии на рынке;
\end{enumerate}

Экономика как наука занимается изучением упомятых выше институтов с целью улучшения их деятельности. Как наука экономика по традиции разделяется на микроэкономику и макроэкономику.

В любом изучаемом экономическом объекте мы будем выделять известные характеристики:
\begin{equation}
x_1, x_2, \cdots, x_k
\end{equation}
, искомые характристики 
\begin{equation}
y_1, y_2, \cdots, y_m
\end{equation}
и взаимосвязи велчин (1) и (2)
\begin{equation}
 F(\vec{y}, \vec{x})
 \end{equation}

Экономика как наука представляет собой сформулированные взаимосвязи наиболее значимых известных и искомых характеристик микро- и макро- экономических объектов.

В методе математического моделирования изучения экономики упомянутые выше взаимосвязи описываются математическим языком и в результате такой записи возникает математическая модель объекта. 

\textbf{Определение.} Экономико-математическая модель (ЭММ) объекта - это некоторое математическое выражение (график или таблица, уравнение или система уравнений, дополненная, возможно, неравенствами, условие экстремума), связывающее воедино известные характеристики объекта (1) и его искомые характеристики (2)

\textbf{Терминология.} Известные характеристики (1) - это экзогенные переменные модели, искомые величины (2) - это эндогенные переменные модели. \\
\begin{center}
ЭКЗОГЕННЫЕ ПЕРЕМЕННЫЕ -> МОДЕЛЬКА -> ЭНДОГЕННЫЕ ПЕРЕМЕННЫЕ
\end{center}

\begin{center}
\textbf{Два класса экономико-математических моделей}
\end{center}

Всё множество математических моделей, математических объектов можно разделить на два класса. В первый класс относятся модели, которые описывают изучаемые объекты такими какими эти объекты являются в реальности модели входящие в этот класс принято называть дескриптивными (описательными) моделями. Вот самый общий вид таких моделей:
\begin{equation}
F(\vec{y}, \vec{x}) = 0;
\end{equation}

Здесь символом $\vec{y}$ обозначен набор эндогенных переменных (2), символом $\vec{x}$ набор экзогенных переменных (1), символом $F$ обозначены записанные математическим языком взаимосвязи величин (1) и (2). Модель (4) задаёт эндогенные переменные $\vec{y}$, как неявные функции экзогенных переменных $\vec{x}$. Выражение (4) это всегда система уравнений (линейные алгебраические, нелинейные, дифференциальные уравнения и возможно интегральные уравнения). Количество уравнений непременно совпадает с количеством эндогенных переменных. Дискриптивные модели.

Во второй класс включаются модели в которых отычкиваются такие значения эндогенных переменных, которые удовлетворяют некоторому требованию оптимальности, вот самый общий вид таких моделей:

\begin{equation}
\begin{cases}
\phi = \phi(\vec{x};\vec{y}) \rightarrow ext (\min, \max),\\
\vec{y} \in Y{\vec{x}}
\end{cases}
\end{equation}

В первой строчке выражения (5) записано требование оптимальности к значиям эндогенных переменных $\vec{y}$. Во втой строчке минимальные требования к эндогенным переменным. Символом $Y$ мы обозначили множество допустимых значений $\vec{y}$ и это множество в общем случае зависит от экзогенных переменных $\vec{x}$. Модели входящие во второй класс принято именовать \textit{оптимизационными} в математике такие модели называются задачами математического программирования на условный экстремум. Функция $\phi$ именуется целевой функцией или иногда критерием. Добавим к сказанному, что выражения (4) и (5) принято называть структурной формой соответсвенно дискриптивные и оптимизационной модели. 

\begin{center}
\textbf{Приведённая форма модели предельные величны и эластичность в экономике.}
\end{center}

Для расчёта по модели (4) или (5) её необходимо трансформировать к приведённой форме: $\vec{y} = f(\vec{x})$. Пример трансформации модели (4) к приведённой форме обсуждён на занятиях (семинар №1 и №2). Приведённая форма модели позволяет получить взаимосвязь заданных изменений экзогенных переменных с возникающими в ответ изменениями эндогенных переменных. 
\begin{equation}
\triangle \vec{y} = f'(\vec{x}) \cdot \triangle \vec{x}
\end{equation}

Символом $f'(\vec{x})$ обозначена матрица частных производных, которая в матиматике называется \textit{матрица Якоби}; её элементы имеют смысл изменений эндогенных перменных в ответ $\triangle \vec{y}$ в ответ на еденичные изменения экзогенных переменных и называются такие элементы п\textit{редельными значениями эндогенных переменных}. 

Проиллюстрируем понятие предельных велечин экономики на примере простейшей модели спроса на некоторое благо.
$$y_t^d = a_0 + a_1 p + a_2  x$$

Коэффициент $a_1$ имеет смысл изменения спроса на данное благо в ответ на повышение цены на одну еденицу. Этот коэффициент носит название \textit{предельного спроса по цене}. Коэффициент $a_2$ имеет смысл изменения спроса на данное благо в ответ на увеличение дохода потреьтителя $x$ на еденицу. Его можно посчитать по следующему правилу

$$M_y(x) = \frac{\partial y}{\partial x} = a_2$$
$$M_y(p) = \frac{\partial y}{\partial p} = a_1$$

Формула (6) подробно выглядит так:

$$\triangle y = a_0 + \frac{\partial y}{\partial p} p + \frac{\partial y}{\partial x} x$$

Матрица $f'(\vec{x)} = \left(\frac{\partial y}{\partial p} \frac{\partial y}{\partial x} \right)^T$

Вектор $\triangle \vec{x} = (\triangle p \triangle x)^T$

Определение эластичности.

По мимо предельных велечин в экономике в процессе анализа объекта методом математического моделирования постоянно используется эластичность эндогенных переменных по экзогенным. Эластичность определяется по следующему правилу:

\begin{equation}
E_{y_i}(x_i) = \frac{\triangle y_i}{y_i} : \frac{\triangle x_j}{x_j}
\end{equation}
является безразмерной велечино, позволяет вычислить относительные изменения эндогенной переменной в ответ на заданное изменение соответствующей экзогенной переменной $\displaystyle{\frac{\triangle x_j}{x_j}}$. Эластичность имеет смысл относительного изменения эндогенной переменной в $\%$ в ответ на относительное изменение экзогенной переменной на $1\%$.

Из определения эластичность можно получить следующуу формулу для её расчёта.

$$E_{y_i}(x_j) = \frac{\triangle y_i}{x_j} : \frac{y_i}{x_j} = M_{y_i}(x_j) : A_{y_i}(x_j)$$

Делитель в правой части имеет среднее значение $\displaystyle{\frac{y_i}{x_j}}$

\textbf{Итог}. При изучении экономического объекта методом математического моделирования создаётся модель одно из двух классов: дискриптивная или оптимизационная. Искомые характеристики объекта и анализ объекта осуществляются при помощи приведённой формы модели.



\end{document}