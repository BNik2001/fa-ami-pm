\documentclass[12pt,a4paper]{article}
\usepackage[14pt]{extsizes}
\usepackage[utf8]{inputenc}
\usepackage{amsfonts}
\usepackage{amssymb}
\usepackage{cmap}
% for fonts
    \usepackage[T2A, T1]{fontenc}
    \usepackage[english, russian]{babel}
    \usepackage{fontspec}
    \defaultfontfeatures{Ligatures=TeX,Renderer=Basic}
    \setmainfont[Ligatures={TeX, Historic}]{Times New Roman}
    \setsansfont{Times New Roman}
    \setmonofont{Courier New}
%plot
%mathcha.io
\usepackage{amsmath}
\usepackage{tikz}
\usepackage{mathdots}
\usepackage{yhmath}
\usepackage{cancel}
\usepackage{color}
\usepackage{siunitx}
\usepackage{array}
\usepackage{multirow}
\usepackage{amssymb}
\usepackage{gensymb}
\usepackage{tabularx}
\usepackage{booktabs}
\usetikzlibrary{fadings}
%mathcha.io
\usepackage{mathrsfs}
\usetikzlibrary{arrows}
\pagestyle{empty}
%plot
\usepackage{float}% for \begin{figure}[H]
\usepackage{cases}
\usepackage{graphicx}
\usepackage[left=2cm,right=2cm,top=2cm,bottom=2cm]{geometry}
\author{Аверьянов Тимофей, Корякин Алексей}
\begin{document}
\begin{center}
\textbf{Лекция №6}

\textbf{Модель Хикса потребления потребителя на рынке благ}

\textbf{План}
\end{center}

\begin{enumerate}
\item  Модель поведения потребителя Хикса в структурной форме и её трансформация к приведённой форме методом Лагранжа (задача математического программирования);
\item  Функция расходов потребителя и её свойства
\item $\displaystyle \boxed{\text{ДЗ}}$
\end{enumerate}
\begin{center}
\textbf{Модель поведения потребителя Хикса}
\end{center}
	В моделе Хикса заложенно следующее утверждение: потребитель выбирает такой набор благ, который с одной стороны имеет наименьшую \textit{стоимость}, а с другой стороны предоставляет потребителю \textit{заданные уровень полезности}.

	Вот математическая запись идей Хикса:


\begin{equation}
\begin{cases}
M\ ={\displaystyle \sum ^{n}_{i=1} p_{i} x_{i}\rightarrow \min}\\
u( x_{1} ,\ \dotsc ,\ x_{n}) \ =u_{0}\\
x_{1} \ \geq 0,\ \dotsc ,\ x_{n} \geq 0
\end{cases}
\end{equation}
Экзогенные переменные:


\begin{equation*}
\vec{p} =( p_{1} ,\ \dotsc ,\ p_{n}) ,\ u_{0} \ -\text{экзогенные переменные}
\end{equation*}
Эндогенные переменные:


\begin{equation*}
\vec{x} =( x_{1} ,\ \dotsc ,\ x_{n}) \ -\ \text{эндогенные переменные}
\end{equation*}


Выражение (1) - это структурная форма модели Хикса. С позиции математики модель (1) - это задача математического программирования на условный экстремум и решать такую задачу можно методом Лагранжа. Метод Лагранжа состоит из следующий шагов:
\begin{enumerate}
\item Составляется функция Лагранжа: $\displaystyle L={\displaystyle \sum _{i} p}_{i} \ x_{i} \ \ +\ l( M\ -\ u( x_{1} ,\ \dotsc ,\ x_{n}))$
\item Cоставляется необходимое условие экстремума:
\end{enumerate}
\begin{equation}
\begin{cases}
\frac{\partial L}{\partial x_{i}} \ =0;\\
\frac{\partial L}{\partial l} \ =\ 0;\\
i\ =\ ( 1,\ 2,\dotsc ,\ n)
\end{cases}
\end{equation}
	3. Эти условия представляют систему $\displaystyle n+1$ уравнений с $\displaystyle n+1$ переменной.

	Система (4) решается либо аналитически, либо численно.


\begin{equation}
\vec{x}^{H} \ =\left( x^{H}_{1} ,\ \dotsc ,\ x^{H}_{n}\right) =\vec{x}^{H}\left(\vec{p} ,\ u_{0}\right)
\end{equation}
\begin{equation}
l^{*} \ =l^{*}\left(\vec{p} ,u_{0}\right)
\end{equation}
\textbf{	Задача}. Пусть пространство благ двухметрно $\displaystyle \vec{x} \ =( x_{1} ,\ x_{2}) \in C\ \sqsubseteq R^{+}_{2}$.

	Пусть функцией полезности потребителя служит логарифм Бернулли в ситуации двух благ эта функция имеет уравнение:


\begin{equation*}
u\ =a_{1} \ ( =\ 0.1) \ \cdotp \ \ln x_{1} \ +\ a_{2}( =\ 0.2) \ \cdotp \ \ln \ x_{2} \
\end{equation*}
Дано:

$\displaystyle \vec{x} \ =\ \left( x_{1} \ \left( =\text{молоко}\right) ,\ x_{\ 2} \ \left( =\text{хлеб}\right)\right)$

$\displaystyle M\ =\ 200$

$\displaystyle p_{1} \ =\ 50\ \text{p/кг}$

$\displaystyle p_{2} \ =\ 75\ \text{p/л}$

На предшействующем семинаре мы отсыскали спрос потребителя по модели Маршала-Вальраса, задавшись значение $\displaystyle M\ =\ 200\ \text{руб.}$. Там же мы рассчитали уровень полезность спроса по Маршалу-Вальрассу:
\begin{equation*}
u\ =\ a_{1} \ \ln x_{1} \ +\ a_{2} \ \ln x_{2} \ =\ 0.1\ \ln \ 1.3\ \ +\ 0.2\ \ln 1.77\ =\ 0.14
\end{equation*}
\textbf{Найти:}

Методом Лагранжа трансформировать к приведённой форме модель Хикса, зная, что $\displaystyle u_{0} \ =\ 0.14$. Вычислить спрос по Хиксу и стоимость спроса по Хиксу ($\displaystyle M^{*} ={\displaystyle \sum ^{2}_{i=1} p_{i} x^{H}_{i}}$(7)).

\textbf{Решение:}

	1. Функция Лагранжа:
\begin{equation*}
L( x_{1} ,\ x_{2} ,\ l) \ =p_{1} x_{1} \ +\ p_{2} x_{2} \ +l( u_{0} \ -\ ( a_{1} \ \cdotp \ \ln x_{1} \ +\ a_{2} \ \cdotp \ \ln \ x_{2}))
\end{equation*}
	2. Cоставляется необходимое условие экстремума:


\begin{equation*}
\begin{cases}
\frac{\partial L}{\partial x_{1}} \ =\ p_{1} \ +l\frac{a_{1}}{x_{1}} \ =\ 0;\\
\frac{\partial L}{\partial x_{2}} \ =\ p_{2} \ +l\frac{a_{2}}{x_{2}} \ =\ 0;\\
\frac{\partial L}{\partial l} =u_{0} -a_{1} \ \cdotp \ \ln x_{1} \ -\ a_{2} \ \cdotp \ \ln \ x_{2} =0;
\end{cases}
\end{equation*}


	3. $\displaystyle \boxed{\text{ДЗ}}$Решить систему и доказать, что решение этой системы методом подстановки имеет следующий вид:
\begin{equation*}
x^{*}_{1} \ =\beta \cdot e^{\frac{u_{0}}{\sum a_{i}}} \cdot p^{-\frac{a_{2}}{\sum a_{i}}}_{1} \cdot p^{\frac{a_{2}}{\sum a_{i}}}_{2} \eqno(3')
\end{equation*}
\begin{equation*}
x^{*}_{2} \ =\ \gamma \cdot e^{\frac{u_{0}}{\sum a_{i}}} \cdot p^{\frac{a_{1}}{\sum a_{i}}}_{1} \cdot p^{-\frac{a_{1}}{\sum a_{i}}}_{2} \eqno(3')
\end{equation*}
\begin{equation*}
l^{*} =\alpha \ \cdot e^{\frac{u_{0}}{\sum a_{i}}} \cdot p^{\frac{a_{2}}{\sum a_{i}}}_{1} \cdot p^{\frac{a_{2}}{\sum a_{i}}}_{2} \eqno(4')
\end{equation*}
Рассчитать численно значение спроса по Хиксу и параметры выше.
\begin{center}
\textbf{Функция рассходов потребителя и её свойства}
\end{center}
Функцией \textbf{рассходов потребителя} можно называть стоимость спроса по Хиксу, как функцию экзогенных переменных модели.


\begin{equation*}
M^{*} ={\displaystyle \sum _{i=1} p_{i} x^{H}_{i} =\ M^{*}\left(\vec{p} ,u_{0}\right)} \eqno(7)
\end{equation*}
Задача. Подстваит правые части из уравнения (3') в выражение (7) и получим следующее уравнение:

$\displaystyle \boxed{\text{ДЗ}}$ Убедиться в правильности формулы (8):
\begin{equation*}
M^{*} \ =\ \psi \cdot e^{\frac{u_{0}}{\sum a_{i}}} \cdot p^{\frac{a_{1}}{\sum a_{i}}}_{1} \cdot p^{\frac{a_{2}}{\sum a_{i}}}_{2} \eqno(8)
\end{equation*}
$\displaystyle \boxed{\text{ДЗ}}$ Проверить справедливость уравнения (8), рассчитать $\displaystyle M^{*} \ $и сравнить полученное значение со значением $\displaystyle M_{0} =200\ \text{оуб.}$.

\textbf{Задача. }Показать, что:

	1. $\displaystyle M^{*} \uparrow u_{0} ;( 9)$

	2. $\displaystyle M^{*} \uparrow p_{i} ;$

\textbf{Решение:}

Докажем (9):
\begin{equation*}
\frac{\partial M^{*}}{\partial u_{0}}  >0
\end{equation*}
\begin{equation*}
M^{*} '_{u_{0}} =\psi \ \cdot \frac{1}{\sum a_{i}} \ \cdot e^{\frac{u_{0}}{\sum a_{i}}} \ \cdot p^{\frac{a_{1}}{\sum a_{i}}}_{1} \cdot p^{\frac{a_{2}}{\sum a_{i}}}_{2}  >0
\end{equation*}
\begin{equation*}
M^{*} '_{p_{i}} \ =\ \psi \cdot e^{\frac{u_{0}}{\sum a_{i}}} \cdot p^{\frac{a_{2}}{\sum a_{i}}}_{2} \cdot \frac{a_{1}}{\sum a_{i}} \cdot p^{\frac{a_{1}}{\sum a_{i}} -1}_{1}
\end{equation*}
Кроме свойств 1 и 2 функция расходов потребителя является выпуклой вверх функцией:
\begin{equation*}
\boxed{\text{ДЗ}}\frac{\partial ^{2} M^{*}}{\partial p^{2}_{i}}  >0
\end{equation*}
\end{document}
