\documentclass[12pt,a4paper]{article}
\usepackage[14pt]{extsizes} 
\usepackage[utf8]{inputenc}
\usepackage{amsmath}
\usepackage{amsfonts}
\usepackage{amssymb}
\usepackage{cmap}
% for fonts
    \usepackage[T2A, T1]{fontenc}
    \usepackage[english, russian]{babel}
    \usepackage{fontspec}
    \defaultfontfeatures{Ligatures=TeX,Renderer=Basic}
    \setmainfont[Ligatures={TeX, Historic}]{Times New Roman}
    \setsansfont{Times New Roman}
    \setmonofont{Courier New}
%plot
\usepackage{pgf,tikz,pgfplots}
\pgfplotsset{compat=1.15}
\usepackage{mathrsfs}
\usetikzlibrary{arrows}
\pagestyle{empty}
%plot
\usepackage{float}% for \begin{figure}[H]
\usepackage{cases}
\usepackage{graphicx}
\usepackage[left=2cm,right=2cm,top=2cm,bottom=2cm]{geometry}
\author{GH-TIMe}
\begin{document}
\begin{center}
\section*{Завершение темы. Эконометрика, её задачи и методы.}
\end{center}
\begin{center}
\textbf{План}
\end{center}
\begin{enumerate}
\item Второй принцип спецификации эконометрических моделей и приведённая форма простейшей макромодели Кейнса.
\item Отражения в спецификации эконометрической модели фактора времени: спецификация динамический моделей или 3-ий принцип спецификации эконометрических моделей, приведённая форма модели, как инструмент анализа изучаемого объекта.
\item Предельные величины в экономике.
\end{enumerate}
На лекции 1 мы получили структурную форму макромодели Кейнса. Для расчётов в данной модели (по моделе Кейнса) необходимо трансформировать его к такому виду в котором каждая эндогенная переменная оказывается выраженной только через объясняющие переменные.
\begin{equation}
\begin{cases}
Y = C + I; \\
C = a_0 + a_1 \cdot Y; \\
0 < a_1 < 1;
\end{cases}
\end{equation}

Чтобы такая трансформация оказалась возможной необходим второй принцип: количество уравнений в структурной форме (1) совпадало с числом эндогенных переменных $\vec{y}$. 

Проиллюстрируем трансформацию к модели Кейнса:
\begin{enumerate}
\item Правую часть первого уравнения структурной формы модели Кейнса, (1) подставим во второе уравнение и выразим из него искомую переменную $C$ через экзогенную переменную, $I$. Получим приведенную форму $C$.
\item Приведенную форму $C$ подставим в первое уравнение (1) и выразим из него $Y$. В итоге получим приведенную форму.
\end{enumerate}
\textbf{Приведённая форма модели Кейнса}
\begin{equation}
\begin{cases}
C = \displaystyle{\frac{a_0}{1-a_1} + \frac{a_1}{1-a_1}} \cdot I; \\
Y = \displaystyle{\frac{a_0}{1-a_1} + \frac{1}{1-a_1}} \cdot I;
\end{cases}
\end{equation}

Чтобы модель можно было трансформировать в приведённую форму число уравнений модели обязано должно совпадать с количеством объясняемых переменных.

\textbf{Третий принцип спецификации модели: отражение фактора времени.}

Фактор времени часто присутствует в условиях экономических задач. Для отражения в модели фактора времени переменные модели датируются. В итоге возникает динамическая модель.

\textbf{Пример.} Задача Линтнера о прогнозе уровня дивидендов: Ч. Ли, Дж. Финнерти Финансы корпораций "ИНФРА-М", 2000, стр. 333)
\begin{itemize}
\item Исходные данные - чистая прибыль на акцию в текущем периоде, $EPS_t$
\item Искомые неизвестные - уровень дивидендов на акцию в том же периоде, $DPS_t$
\end{itemize}

Взаимосвязи между величиной $EPS$ и $DPS$ сформулированы в следующих двух утверждениях (\textit{Задача Линтнера о прогнозе уровня дивидендов}):
\begin{enumerate}
\item Фирма обладает долгосрочной целевой долей текущей прибыли ($\gamma$), которую она желает выплачивать в качестве дивидендов своим акционерам в текущем периоде.
\item Реальный уровень дивидендов в текущем периоде, $DPS$, определяется:
	\begin{enumerate}
	\item желаемым уровнем дивидендов в текущем периоде ($DPS^w$)
	\item реальным уровнем дивидендов в предшествующем периоде ($DPS_{t-1}$)
	\end{enumerate}
\end{enumerate}

Запишем эти утверждения математическим языком, обращая внимание, что во втором утверждении мы обязаны сделать различия между дивидендами в текущем периоде, а это означает, что во втором утверждении содержится \textit{фактор времени}. 
\begin{equation}
\begin{cases}
DPS_t^w = \gamma \cdot EPS_t; \\
DPS_t = \lambda DPS_t^w + (1 - \lambda) DPS_{t-1} \\
0 \leqslant \gamma \leqslant 1, \quad 0 \leqslant \lambda \leqslant 1
\end{cases}
\end{equation}
\textit{$t$ -- время; $\lambda$ -- коэффициент корректировки.}

\textbf{Комментарий.} \textit{В первой строчке в левой части присутствует желаемый $DPS^w$ это не наблюдаемая переменная, но она появилась в модели согласно этому утверждению и эта переменная рассматривается, как эндогенная. Во второй строчке в качестве линейной функции двух переменных $DPS^w$ и $DPS$ принято линейная однородная функция с положительными коэффициентами сумма которых равна 1. Добавим к сказанному, что уровень дивидендов это нечто среднее между желаемым уровнем дивидендов ($DPS^w$) и реальными дивидендами ($DPS_t$) в предшествующем периоде. Данная запись это структурная форма задачи Линтнера и в этой форме все переменные датированы (привязаны ко времени).}

\textbf{Типы переменных в динамических моделях.}\\
Объясняемые переменные в динамических моделях принято принято называть текущими эндогенными переменными; в модели Линтнера их две: $DPS_t^w, DPS_t$. \\
Объясняющие могут включать в себя:
\begin{enumerate}
\item Лаговые эндогенные переменные, $DPS_{t-1}$
\item Лаговые экзогенные переменные, (смотри Семинар №2)
\item Текущие экзогенные переменные, $EPS_t$
\end{enumerate}
\framebox[1.1\width]{Д/з} Трансформировать модель Линтнера к приведённой форме. При отражении фактора времени возникает динамическая модель в которой объясняемые переменные. Приведённая форма, как инструмент анализа экономического объекта, задачи.

Вернёмся к модели Кейнса. И построим график функции $C$ от объёма инвестиций $I$:

\begin{figure}[H]
\begin{center}
\begin{tikzpicture}[line cap=round,line join=round,>=triangle 45,x=1.0cm,y=1.0cm]
\begin{axis}[
xlabel={$I$},
ylabel={$C$},
axis lines=center,
ymajorgrids=true,
xmajorgrids=true,
x=1.0cm,y=1.0cm,
xticklabels={,,},
yticklabels={,,},
axis lines=middle,
xmin=0.0,
xmax=8.0,
ymin=0.0,
ymax=5.0,
xtick={0.0,1.0,...,7.0},
ytick={0.0,1.0,...,4.0}]
\clip(0.,0.) rectangle (7.,4.);
\draw [line width=2.pt,domain=0.0:7.0] plot(\x,{(--3.44--1.42*\x)/3.44});
\draw [line width=2.pt] (1.9909776173285199,1.8218570397111913)-- (4.,1.82);
\draw [line width=2.pt] (4.,1.82)-- (3.989009097472924,2.646625848375451);
\draw (2,1.8) node[anchor=north west] {$\triangle l = 1$};
\draw (4,2.94) node[anchor=north west] {$\triangle C = \displaystyle{\frac{a_1}{1 - a_1}}$};
\draw (0.16, 3.7) node[anchor=north west] {$\triangle C = \displaystyle{\frac{a_1}{1-a_1} \triangle I}$};
\end{axis}
\end{tikzpicture}
\caption{Измение $C$ от изменения инвестиций}
\end{center}
\end{figure}

Из первого уравнения приведённой формы можно найти взаимосвязь дополнительных инвестиций в экономику $\triangle C$. Получается, что коэффициент возникает в ответ на дополнительную инвестицию $\triangle I$. Построим график. Экономисты называют этот коэффициент придельным уровнем потребления по объёму инвестиций.

\begin{figure}[H]
\begin{center}
\begin{tikzpicture}[line cap=round,line join=round,>=triangle 45,x=1.0cm,y=1.0cm]
\begin{axis}[
x=1.0cm,y=1.0cm,
xlabel={$I$},
ylabel={$Y$},
axis lines=center,
ymajorgrids=true,
xmajorgrids=true,
xticklabels={,,},
yticklabels={,,},
xmin=0.0,
xmax=8.0,
ymin=0.0,
ymax=5.0,
xtick={0.0,1.0,...,7.0},
ytick={0.0,1.0,...,4.0},]
\clip(0.,0.) rectangle (7.,4.);
\draw [line width=2.pt,domain=0.0:7.0] plot(\x,{(--3.3--1.68*\x)/3.3});
\draw [line width=2.pt] (1.1317668679443422,1.5761722236807563)-- (3.3,1.6);
\draw [line width=2.pt] (3.3,1.6)-- (3.3,2.68);
\draw (1.6,1.64) node[anchor=north west] {$\triangle l = 1$};
\draw (3.3,2.84) node[anchor=north west] {$\triangle Y = \displaystyle{\frac{1}{1-a_1}}$};
\draw (0,4) node[anchor=north west] {$\triangle Y = \displaystyle{\frac{1}{1-a_1} \triangle I}$};
\end{axis}
\end{tikzpicture}
\caption{Изменение дохода в ответ на изменение инвестиций}
\end{center}
\end{figure}

Взаимосвязь представлена уравнением:
\begin{equation}
\triangle Y = \displaystyle{\frac{1}{1-a_1} \triangle I}
\end{equation}

\framebox[1.1\width]{Д/з}. Проанализировать знак $\displaystyle{\frac{1}{1-a_1}} (+)$ вычислить значение при $a_1 = 0,6$, которые имеют название \textit{мультипликатора инвестиций Кейнса} и дать экономическую трактовку этого коэффициента. 

Обратим ещё раз внимание, что именно формулы $\triangle C = \displaystyle{\frac{a_1}{1-a_1} \triangle I}$ и $\triangle Y = \displaystyle{\frac{1}{1-a_1} \triangle I}$ действительно позволяют оценить отражения на уровне дохода и инвестиций в стране.

\textbf{Вывод.} Приведённая форма модели служит инструментом, как прогнозирования, так и анализа объекта. В линейных моделях коэффициенты в приведённой форме имеют смысл придельных величин в экономике.

\end{document}