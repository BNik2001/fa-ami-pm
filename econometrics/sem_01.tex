\documentclass[12pt,a4paper]{article}
\usepackage[14pt]{extsizes}
\usepackage[utf8]{inputenc}
\usepackage{amsmath}
\usepackage{amsfonts}
\usepackage{amssymb}
\usepackage{cmap}
% for fonts
    \usepackage[T2A, T1]{fontenc}
    \usepackage[english, russian]{babel}
    \usepackage{fontspec}
    \defaultfontfeatures{Ligatures=TeX,Renderer=Basic}
    \setmainfont[Ligatures={TeX, Historic}]{Times New Roman}
    \setsansfont{Times New Roman}
    \setmonofont{Courier New}
%plot
\usepackage{pgf,tikz,pgfplots}
\pgfplotsset{compat=1.15}
\usepackage{mathrsfs}
\usetikzlibrary{arrows}
\pagestyle{empty}
%plot
\usepackage{float}% for \begin{figure}[H]
\usepackage{cases}
\usepackage{graphicx}
\usepackage[left=2cm,right=2cm,top=2cm,bottom=2cm]{geometry}
\author{Аверьянов Тимофей, Корякин Алексей}
\begin{document}
\begin{center}
\section*{Семинар №1 \\
Два принципа спецификации эконометрических моделей и две их формы}
\end{center}
\begin{center}
\textbf{План}
\end{center}
\begin{enumerate}
\item Первый принцип спецификации экономической модели и ее структурная форма.
\item Второй принцип спецификации эконометрических моделей и приведенная форма моделей. Расчётная схема задач.
\end{enumerate}
\begin{center}
\end{center}

У любого изучаемого экономического объекта имеются известные количественные характеристики и искомые количественные характеристики. Математическая модель объекта — запись математическом языком взаимосвязей известных и искомых характеристик объектов. Модель нужна для искомых характеристик и .. Для расчётных.
В процессе записи математическом языком взаимосвязей известных и искомых характеристик стараются привлекать, прежде всего, линейные алгебраические функции. Известные характеристики объекта называют экзогенными переменными, искомые — эндогенными.

\textbf{Задача 1.}
Изучаемым объектом называется конкурентный рынок некоторого блага. Искомыми характеристиками данного объекта являются:
\begin{itemize}
	\item Уровень спроса данного блага (demand) - $y^d$
	\item Уровень предложения блага (supply) - $y^s$
	\item Цена блага (price) - $p$


В обсуждаемой задаче величины являются эндогенными. Известной характеристикой в данной задаче будет служить душевой доход потребителя. Обозначим символом $Х$.
Между спросом и предложением существует имеются объективно существующие взаимосвязи, которые можно сформулировать следующим образом:
\begin{enumerate}
  \item Уровень спроса объясняется его ценой и душевым доходом. С ростом цены спрос снижается для нормальных товаров. С ростом дохода потребителя спрос возрастает. Такое благо называется ценным.
  \item Закон предложения. Уровень предложения блага объясняется его ценой и с ростом цены предложение увеличивается.
  \item Рыночная цена блага формируется при балансе спроса и предложения.
\end{enumerate}

Теперь необходимо математическом языком записать данные утверждения. Таким способом мы получим структурную форму простейшей модели спроса-предложения.
(1) означает что переменная $y^d$ - функция переменных $P$ и $X$.
$y^d=y^d (P, X)$

Воспользовавшись первым принципом спецификации мы выберем подходящую линейную функцию аргументов $P$ и $X$.
\begin{equation}
\begin{cases}
y^d= a_0+a_1p+a_2x; \\
a_1 < 0; \\
a_2 > 0;
\end{cases}
\end{equation}
(3) - простейшая функция спроса
$y^s=y^s(p)$

\begin{equation}
\begin{cases}
y^s=a_3+a_4p; \\
a_2 > 0; % ?????
\end{cases}
\end{equation}

Структурная форма простейшей модели спроса-предложения нормального ценного блага на конкурентном рынке.
В структурной форме данной модели искомые величины и известная величина тесно переплетены. Для расчета по модели ее нужно трансформировать к такому виду, где каждая искомая величина будет выражена только через известную величину $Х$.
\begin{equation}
\begin{cases}
y^d=y^d(x); \\
y^s=y^s(x) \\
P=P(x);
\end{cases}
\end{equation}

Второй принцип спецификации эконометрических моделей и приведенная форма модели.
Принцип требует чтобы количество уравнений совпадало с количеством искомых переменных. Является необходимым условием для представления каждой искомой величины.
Это служит и является необходимым условием для представления искомой величины в виде явной функции известных характеристик (экзогенных переменных)

\framebox[1.1\width]{Д/з}. 	Проверить, что наши шаги корректные (нет деления на ноль). Найти экономическое представление.

\end{document}
