\documentclass[12pt,a4paper]{article}
\usepackage[14pt]{extsizes}
\usepackage[utf8]{inputenc}
\usepackage{amsmath}
\usepackage{amsfonts}
\usepackage{amssymb}
\usepackage{cmap}
% for fonts
    \usepackage[T2A, T1]{fontenc}
    \usepackage[english, russian]{babel}
    \usepackage{fontspec}
    \defaultfontfeatures{Ligatures=TeX,Renderer=Basic}
    \setmainfont[Ligatures={TeX, Historic}]{Times New Roman}
    \setsansfont{Times New Roman}
    \setmonofont{Courier New}
\usepackage{pgfplots} % plot
% mathcha
\usepackage{tikz}
\usepackage{mathdots}
\usepackage{yhmath}
\usepackage{cancel}
\usepackage{color}
\usepackage{siunitx}
\usepackage{array}
\usepackage{multirow}
\usepackage{amssymb}
\usepackage{gensymb}
\usepackage{tabularx}
\usepackage{booktabs}
\usetikzlibrary{fadings}
% mathcha
\usepackage{adjustbox} % for table size
\usepackage{float} % for H figure
\pgfplotsset{compat=1.15}
\usepackage{graphicx}
\usepackage[left=2cm,right=2cm,top=2cm,bottom=2cm]{geometry}
\author{GH-TIMe, little KoryakinAK}
\begin{document}
\begin{center}
\textbf{План}
\end{center}

\begin{enumerate}
\item Функция потребления Кейнса и реальные данные;
\item Общий вид эконометрической модели с отражённым влиянием на эндогенные переменные неучтённых факторов;
\item Временной ряд и структура его уровней;
\end{enumerate}

	На прошлой лекции обсудили отражение в моделе фактора времени и использование модели, как инструмента анализа изучаемого объекта. 

	На сегодняшней лекции мы исследуем соответсвие математических моделей реальным данным и научимся отражать в моделе воздействие на искомые характреристики объекта (на текущие эндогенные переменные) неучтённых факторов. Наши исследования мы проведём на простейшей макромодели Кейнса:

\begin{equation}
\begin{cases}
Y\ =\ C+I; & \\
C\ =\ a_{0} \ +\ a_{1} \ \cdot Y; & 0\ < \ a_{1} \ < \ 1;
\end{cases}
\end{equation}

	Согласуется ли эта функция с реальной статистикой?

	Исследование проведём по следующей схеме:

На плоскости зададим декартову систему координат и по оси обцисс отложим содержащиеся в табл.1 уровни ВВП РФ, на оси ординат отложим соотсветсвующие значения уравнений потребления; Если модель Кейнса соответсвует реальным данным, то точки графика расположатся на восходящей прямой. 


\begin{table}[!h]
\centering
\begin{tabular}{|c|c|c|c|c|c|c|}
\hline 
 \textbf{Год} & \textbf{2003} & \textbf{2004} & \textbf{2005} & \textbf{2006} & \textbf{2007} & \textbf{2008} \\
\hline 
$\displaystyle \textbf{Y}$ & 6410 & 7288 & 8196 & 8915 & 10002 & 10767 \\
\hline 
$\displaystyle \textbf{C}$ & 4911 & 5554 & 6290 & 6739 & 7305 & 7773 \\
\hline 
$\displaystyle \textbf{I}$ & 1499 & 1734 & 1906 & 2175 & 2995 & 2994 \\
 \hline
\end{tabular}
\caption{Статистические данные}
\end{table}


\begin{figure}[H]
\begin{center}
\tikzset{every picture/.style={line width=0.75pt}} %set default line width to 0.75pt        

\begin{tikzpicture}[x=0.75pt,y=0.75pt,yscale=-1,xscale=1]
%uncomment if require: \path (0,312.8249969482422); %set diagram left start at 0, and has height of 312.8249969482422

%Shape: Axis 2D [id:dp8743509274123296] 
\draw  (50,251.98) -- (384.3,251.98)(83.43,68) -- (83.43,272.43) (377.3,246.98) -- (384.3,251.98) -- (377.3,256.98) (78.43,75) -- (83.43,68) -- (88.43,75)  ;
%Straight Lines [id:da44467330718849607] 
\draw    (81.3,206.43) -- (347.3,123.43) ;

%Straight Lines [id:da6441726223509632] 
\draw    (140.3,187.43) -- (140,230) ;
\draw [shift={(140,230)}, rotate = 90.4] [color={rgb, 255:red, 0; green, 0; blue, 0 }  ][fill={rgb, 255:red, 0; green, 0; blue, 0 }  ][line width=0.75]      (0, 0) circle [x radius= 3.35, y radius= 3.35]   ;

%Straight Lines [id:da45408138515545904] 
\draw    (179.3,134.43) -- (179,176) ;

\draw [shift={(179.3,134.43)}, rotate = 90.41] [color={rgb, 255:red, 0; green, 0; blue, 0 }  ][fill={rgb, 255:red, 0; green, 0; blue, 0 }  ][line width=0.75]      (0, 0) circle [x radius= 3.35, y radius= 3.35]   ;

%Straight Lines [id:da3675690211779734] 
\draw    (229.3,159.43) -- (229,201) ;
\draw [shift={(229,201)}, rotate = 90.41] [color={rgb, 255:red, 0; green, 0; blue, 0 }  ][fill={rgb, 255:red, 0; green, 0; blue, 0 }  ][line width=0.75]      (0, 0) circle [x radius= 3.35, y radius= 3.35]   ;

%Straight Lines [id:da8370177312160028] 
\draw    (259.3,131.43) -- (259.3,151.43) ;

\draw [shift={(259.3,131.43)}, rotate = 90] [color={rgb, 255:red, 0; green, 0; blue, 0 }  ][fill={rgb, 255:red, 0; green, 0; blue, 0 }  ][line width=0.75]      (0, 0) circle [x radius= 3.35, y radius= 3.35]   ;
%Straight Lines [id:da37313575108824626] 
\draw    (290.3,99.43) -- (290,141) ;

\draw [shift={(290.3,99.43)}, rotate = 90.41] [color={rgb, 255:red, 0; green, 0; blue, 0 }  ][fill={rgb, 255:red, 0; green, 0; blue, 0 }  ][line width=0.75]      (0, 0) circle [x radius= 3.35, y radius= 3.35]   ;
%Straight Lines [id:da6953659916864847] 
\draw    (330.3,128.43) -- (330.3,152.43) ;
\draw [shift={(330.3,152.43)}, rotate = 90] [color={rgb, 255:red, 0; green, 0; blue, 0 }  ][fill={rgb, 255:red, 0; green, 0; blue, 0 }  ][line width=0.75]      (0, 0) circle [x radius= 3.35, y radius= 3.35]   ;


% Text Node
\draw (112,213) node  [align=left] {$\displaystyle u_{1}$};
% Text Node
\draw (149,146) node  [align=left] {$\displaystyle u_{2}$};
% Text Node
\draw (210,194) node  [align=left] {$\displaystyle u_{3}$};
% Text Node
\draw (244,122) node  [align=left] {$\displaystyle u_{4}$};
% Text Node
\draw (312,101) node  [align=left] {$\displaystyle u_{5}$};
% Text Node
\draw (363,153) node  [align=left] {$\displaystyle u_{6}$};
% Text Node
\draw (99,75) node  [align=left] {$\displaystyle C$};
% Text Node
\draw (363,233) node  [align=left] {$\displaystyle Y$};


\end{tikzpicture}
\end{center}
\caption{Уравнение потребления}
\end{figure}

Рассмотрев построенный график, делаем следующие выводы:
\begin{enumerate}
\item Точки реальных данных (вот эти ромбики) не расположены на восходящей прямой, и это значит, что модель Кейнса в полной мере не соответствует реальным данным (не соответствует изучаемому объекту). Причина несоответствия -- воздействие на совокупное потребление в стране неучтенных факторов.$\displaystyle \boxed{\text{ДЗ}}$ Сформулировать факторы, которые оказывают воздействие на совокуное потребление в стране и отсутсвует в модели Кейнса. 
\item Точки реальных данных расположены вдоль ощущаемой восходящей прямой. Это значит, что модель Кейнса правильно отражает тенденцию, согласно которой изменяется совокупное потребление в стране в ответ на визменение дохода. Модель Кейнса не улавливает всех изменений совокупного потребления в стране, вызванныех неуточнёнными факторами, но правильно отражен главный фактор потребления - доход.
\item Точки реальных данных хаотично разбросаны вдоль восходящей прямой.
\end{enumerate}

	На основании п.1-3 можем предположить аналитическое описание диаграммы:
\begin{equation}
C\ =\ a_{0} \ +\ a_{1} \ \cdot Y\ +\ u( uncertain)
\end{equation}
,где $\displaystyle u$ - переменная велечина, которая принимает то положительное, то отрицательное значение рассеянное вокруг нуля. В силу хаотичности появления её значений экономисты называют \underline{случайным возмущением}. Физики и в технических приложениях такие величины называются невязками или ошибками модели.

	Основные характеристики случайного возмущения:
\begin{enumerate}
\item $\displaystyle E( u) =0$ - среднее значение $\displaystyle u$, равное 0;
\item $\displaystyle E\left( u^{2}\right) \ =\sigma ^{2}_{u}$, где $\displaystyle \sigma _{u}$ - мера влияния неучтённых факторов; $\displaystyle \sigma ^{2}_{u}$ средний квадрат разброса значений случайных возмущений вокруг мат. ожидания;
\end{enumerate}

	Отсюда следует спецификация эконометрической модели Кейнса в которой отражено влияние на $\displaystyle C$ неуточенных факторов:
\begin{equation}
\begin{cases}
Y\ =\ C+I;\\
C\ =\ a_{0} \ +\ a_{1} \ \cdot Y\ +\ u;\\
0\ < \ a_{1} \ < \ 1;\\
E( u) \ =\ 0,\ E\left( u^{2}\right) \ =\ \sigma ^{2}_{u};
\end{cases}
\end{equation}

Эконометричеными или регрессионными моделями называются дескриптивные ЭММ со случайными возмущениями в поведенческих уравнениях.

	Приведём спецификацию эконометрической модели интерна
\begin{equation}
\begin{cases}
DPS^{e}_{t} \ =\gamma \ \cdot EPS_{t} ;\\
0\ \leq \gamma \ \leq 1;\ 0\ \leq \lambda \ \leq 1;\\
DPS_{t} \ =\ \lambda \ \cdot \ DPS^{e}_{t} \ +\ ( 1-\lambda ) \ \cdot DPS_{t-1} \ +\ v_{t} ;\\
E( v_{t}) \ =\ 0,\ E\left( v^{2}_{t}\right) \ =\sigma ^{2}_{v} ;\ 
\end{cases}
\end{equation}

Общий вид эконометрической модели в структурной форме:
\begin{equation}
F\left(\overrightarrow{y_{t}} ,\overrightarrow{x_{t}}\right) =\overrightarrow{u_{t}}
\end{equation}

Структурная форма эконометрической модели из линейной алгебры уравнений:
\begin{equation}
A\ \cdotp \ \overrightarrow{y_{t}} \ +\ B\cdotp \overrightarrow{x_{t}} \ =\ \overrightarrow{u_{t}}
\end{equation}
$\displaystyle \overrightarrow{u_{t}}$ - вектор случайных возмущений, некоторые компоненты могут равняться 0.

	В ситуации Линтерна $\displaystyle \overrightarrow{u_{t}}$ состоит из двух компонент: $\displaystyle \overrightarrow{u_{t}} \ =\ ( 0,\ v_{t})$.

	\textbf{Задача:} Найти $\displaystyle \overrightarrow{y_{t}}$ в моделе Кейнса. $\displaystyle \overrightarrow{x_{t}}$, $\displaystyle \overrightarrow{u_{t}}$? 

	Для отражения в деструктивной модели влияния на объясняемые переменные неучтённых факторов в правых частях поведенческих уранений включаются случайные возмущения; случайные возмущения - та часть эндогенной переменной, которая порождена неуточнёнными факторами.

	Приведенная форма эконометрической модели:
\begin{equation}
\vec{y}_{t} \ =f\left(\vec{x}_{t} ,\overrightarrow{u_{t}}\right)
\end{equation}

	Приведённая форма линейной эконометрической модели:
\begin{equation}
\overrightarrow{y_{t} \ } \ =M\ \cdotp \ \overrightarrow{x_{t}} \ +\ \overrightarrow{\varepsilon _{t}}
\end{equation}
\begin{equation}
E\left( \vartriangle \overrightarrow{y_{t}}\right) \ =\ M\ \cdotp \vartriangle \overrightarrow{x_{t}}
\end{equation}

\textbf{Задача:}Трансфорсировать (3) к приведённой форме.
\end{document}



