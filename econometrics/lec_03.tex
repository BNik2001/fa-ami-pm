\documentclass[12pt,a4paper]{article}
\usepackage[14pt]{extsizes}
\usepackage[utf8]{inputenc}
\usepackage{amsmath}
\usepackage{amsfonts}
\usepackage{amssymb}
\usepackage{cmap}
\usepackage{tikz}
\usepackage{mathdots}
\usepackage{yhmath}
\usepackage{cancel}
\usepackage{color}
\usepackage{siunitx}
\usepackage{array}
\usepackage{multirow}
\usepackage{amssymb}
\usepackage{gensymb}
\usepackage{tabularx}
\usepackage{booktabs}
\usetikzlibrary{fadings}
% for fonts
    \usepackage[T2A, T1]{fontenc}
    \usepackage[english, russian]{babel}
    \usepackage{fontspec}
    \defaultfontfeatures{Ligatures=TeX,Renderer=Basic}
    \setmainfont[Ligatures={TeX, Historic}]{Times New Roman}
    \setsansfont{Times New Roman}
    \setmonofont{Courier New}
\usepackage{pgfplots} % plot
\usepackage{cases}
\usepackage{float}
\pgfplotsset{compat=1.15}
\usepackage{graphicx}
\usepackage[left=2cm,right=2cm,top=2cm,bottom=2cm]{geometry}
\author{GH-TIMe, KoryakinAK}
\begin{document}
\begin{center}
\section*{заголовок (Лекция №3)}
\end{center}

\begin{enumerate}
\item Функция потребления Кейнса и реальные данные.
\item Общий вид эконометрической модели с отражённым влиянием на эндогенные переменные не учтённых факторов
\item Временной ряд и структура его уровней.
\end{enumerate}

	На прошлой лекции обсудили отражение в моделе фактора времени и использование модели, как инструмента анализа изучаемого объекта. На сегодняшней лекции мы исследуем соответсвие математических моделей реальным данным и научимся отражать в моделе воздействие на искомые характреристики объекта (на текущие эндогенные переменные) неучтённых факторов. Наши исследования мы проведём на простейшей макромодели Кейнса


\begin{equation*}
\begin{cases}
Y\ =\ C+I; & \\
C\ =\ a_{0} \ +\ a_{1} \ \cdot Y; & 0\ < \ a_{1} \ < \ 1
\end{cases}
\end{equation*}

Нам предстоит выяснить, согласуется ли эта функция с реальной статистикой, собранной из системы национальных счетов России в таблице 1.

\begin{tabular}{|p{0.14\textwidth}|p{0.14\textwidth}|p{0.14\textwidth}|p{0.14\textwidth}|p{0.14\textwidth}|p{0.14\textwidth}|p{0.14\textwidth}|}
\hline
 Год & 2003 & 2004 & 2005 & 2006 & 2007 & 2008 \\
\hline
 $\displaystyle Y$ & 6410 & 7288 & 8196 & 8915 & 10002 & 10767 \\
\hline
 $\displaystyle C$ & 4911 & 5554 & 6290 & 6739 & 7305 & 7773 \\
\hline
 $\displaystyle I$ & 1499 & 1734 & 1906 & 2175 & 2995 & 2994 \\
 \hline
\end{tabular}

\tikzset{every picture/.style={line width=0.75pt}} %set default line width to 0.75pt

\begin{figure}[H]
\begin{center}
\begin{tikzpicture}[x=0.75pt,y=0.75pt,yscale=-1,xscale=1]
%uncomment if require: \path (0,300); %set diagram left start at 0, and has height of 300

%Shape: Axis 2D [id:dp8743509274123296]
\draw  (50,251.98) -- (384.3,251.98)(83.43,68) -- (83.43,272.43) (377.3,246.98) -- (384.3,251.98) -- (377.3,256.98) (78.43,75) -- (83.43,68) -- (88.43,75)  ;
%Straight Lines [id:da44467330718849607]
\draw    (81.3,206.43) -- (347.3,123.43) ;


%Straight Lines [id:da6441726223509632]
\draw    (140.3,187.43) -- (140,230) ;
\draw [shift={(140,230)}, rotate = 90.4] [color={rgb, 255:red, 0; green, 0; blue, 0 }  ][fill={rgb, 255:red, 0; green, 0; blue, 0 }  ][line width=0.75]      (0, 0) circle [x radius= 3.35, y radius= 3.35]   ;

%Straight Lines [id:da45408138515545904]
\draw    (179.3,134.43) -- (179,176) ;

\draw [shift={(179.3,134.43)}, rotate = 90.41] [color={rgb, 255:red, 0; green, 0; blue, 0 }  ][fill={rgb, 255:red, 0; green, 0; blue, 0 }  ][line width=0.75]      (0, 0) circle [x radius= 3.35, y radius= 3.35]   ;
%Straight Lines [id:da3675690211779734]
\draw    (229.3,159.43) -- (229,201) ;
\draw [shift={(229,201)}, rotate = 90.41] [color={rgb, 255:red, 0; green, 0; blue, 0 }  ][fill={rgb, 255:red, 0; green, 0; blue, 0 }  ][line width=0.75]      (0, 0) circle [x radius= 3.35, y radius= 3.35]   ;

%Straight Lines [id:da8370177312160028]
\draw    (259.3,131.43) -- (259.3,151.43) ;

\draw [shift={(259.3,131.43)}, rotate = 90] [color={rgb, 255:red, 0; green, 0; blue, 0 }  ][fill={rgb, 255:red, 0; green, 0; blue, 0 }  ][line width=0.75]      (0, 0) circle [x radius= 3.35, y radius= 3.35]   ;
%Straight Lines [id:da37313575108824626]
\draw    (290.3,99.43) -- (290,141) ;

\draw [shift={(290.3,99.43)}, rotate = 90.41] [color={rgb, 255:red, 0; green, 0; blue, 0 }  ][fill={rgb, 255:red, 0; green, 0; blue, 0 }  ][line width=0.75]      (0, 0) circle [x radius= 3.35, y radius= 3.35]   ;
%Straight Lines [id:da6953659916864847]
\draw    (330.3,128.43) -- (330.3,152.43) ;
\draw [shift={(330.3,152.43)}, rotate = 90] [color={rgb, 255:red, 0; green, 0; blue, 0 }  ][fill={rgb, 255:red, 0; green, 0; blue, 0 }  ][line width=0.75]      (0, 0) circle [x radius= 3.35, y radius= 3.35]   ;


% Text Node
\draw (112,213) node  [align=left] {$\displaystyle u_{1}$};
% Text Node
\draw (149,146) node  [align=left] {$\displaystyle u_{2}$};
% Text Node
\draw (210,194) node  [align=left] {$\displaystyle u_{3}$};
% Text Node
\draw (244,122) node  [align=left] {$\displaystyle u_{4}$};
% Text Node
\draw (312,101) node  [align=left] {$\displaystyle u_{5}$};
% Text Node
\draw (363,153) node  [align=left] {$\displaystyle u_{6}$};
% Text Node
\draw (99,75) node  [align=left] {$\displaystyle C$};
% Text Node
\draw (363,233) node  [align=left] {$\displaystyle Y$};


\end{tikzpicture}
\end{center}
\caption{Диаграмма рассеивания}
\end{figure}

Наше исследование мы проведем по следующей схеме. На плоскости зададим декартову систему координат и по оси абсцисс отложим содержащиеся в таблице 1 уровни ВВП России, вдоль вертикальной оси отложим соответствующие им значения совокупного потребления в стране. Если модель Кейнса в полной мере соответствует реальным данным, то точки разместятся строго на восходящей прямой. Картина оказывается следующей. Рассматривая этот график (он называется диаграммой рассеивания), можно сделать следующие выводы:
\begin{enumerate}
\item Точки реальных данных (вот эти ромбики) не расположены на восходящей прямой, и это значит, что модель Кейнса в полной мере не соответствует реальным данным (не соответствует изучаемому объекту). Причина несоответствия -- воздействие на совокупное потребление в стране неучтенных факторов.
ДЗ сформулировать факторы, которые в "во всех веротностях оказывают воздействие на совокупное потребление в стране и которые отсутвуют в моделе Кейнса"
\item Это значит, что модель Кейнса правильно отражает тенденцию, согласно которой изменяется потребление в стране в ответ на изменение дохода.
Модель Кейнса не улавливает всех изменений переменной $C$, вызванных неучтенными факторами, но правильно отражает воздействие на переменную $C$ главного фактора потребления -- дохода.
Это значит, что модель Кейнса правильно отражает тенденцию, согласно которой изменяется потребление в стране в ответ на изменение дохода.
Модель Кейнса не улавливает всех изменений переменной $C$, вызванных неучтенными факторами, но правильно отражает воздействие на переменную $C$ главного фактора потребления -- дохода.
\item Точки реальных данных хаотично разбросаны вдоль восходящей прямой.
\end{enumerate}

На основании сделанных выводов мы можем предложить следующее аналитическое описание этой диаграммы.

\begin{equation*}
С\ =\ a_{0} \ +\ a_{1} \ \cdot \ Y\ +\ u
\end{equation*}

Символом $u$ мы обозначаем переменную величину, которая хаотично принимает то положительные, то отрицательные значения, рассеиянные вокруг нуля.
$u$ в силу хаотичного характера появления ее значений экономисты называют случайным возмущением; в физике такие возмущение называют невязками или \textit{ошибками модели}. Мы будет называть их \textit{случайными возмущениями}.
\begin{center}
\textbf{Основные характеристики случайного возмущения (случайной переменной)}
\end{center}


\begin{equation*}
\begin{cases}
Y\ =\ C+I;\\
C\ =\ a_{0} \ +\ a_{1} \ \cdot Y\ +\ u;\\
0\ < \ a_{1} \ < \ 1\\
E( u) \ =\ 0,\ E\left( u^{2}\right) \ =\ \sigma ^{2}_{u}
\end{cases}
\end{equation*}

У случайной перемменной имеются две важные для практики числовые характеристики:
\begin{enumerate}
\item Математическое ожидание (среднее значение случайной переменной) $E(u)$. По предположению среднее значение предполагается равным нулю.
\item Дисперсия. Так называют константу, которая равна среднему квадрату разрбоса значений случайной переменной вокруг математического ожидания ($\sigma^2_u$)
\end{enumerate}
Подводя итог мы можем составить спецификацию эконометрической модели Кейнса, в которой отражено влияение на уровень совокупного потребления неучтенных факторов. Эконометрическими или регрессионными моделями называются дескриптивные эконометрические модели со случайными возмущениями в поведенческих уравнениях.
Приведем спецификацию эконометрчиеской модели Линтнера
случайные возмущения
И наконец отметим самый общий вид эконометрической модели в структурной форме.
Записанные математическом языком взаимосвязи 
\end{document}
