\documentclass[12pt,a4paper]{article}
\usepackage[14pt]{extsizes}
\usepackage[utf8]{inputenc}
\usepackage{amsmath}
\usepackage{amsfonts}
\usepackage{amssymb}
\usepackage{cmap}
% for fonts
    \usepackage[T2A, T1]{fontenc}
    \usepackage[english, russian]{babel}
    \usepackage{fontspec}
    \defaultfontfeatures{Ligatures=TeX,Renderer=Basic}
    \setmainfont[Ligatures={TeX, Historic}]{Times New Roman}
    \setsansfont{Times New Roman}
    \setmonofont{Courier New}
\usepackage{pgfplots} % plot
\usepackage{cases}
\pgfplotsset{compat=1.15}
\usepackage{graphicx}
\usepackage[left=2cm,right=2cm,top=2cm,bottom=2cm]{geometry}
\author{Аверьянов Тимофей, Корякин Алексей}
\begin{document}
\begin{center}
\section*{Третий принцип спецификации экономических моделей: отражение в модели, факторы времени(Семинар №2)}
\end{center}
\begin{center}
\textbf{План}
\end{center}
\begin{enumerate}
\item Спецификация динамической модели спроса-предложения на конкурентом рынке. Типы переменных в динамических моделях.
\item Трансформация динамической модели к приведенной форме. Предельные величины в экономике.
\item ДЗ, защита ДЗ
\end{enumerate}

Мы обсудили два принципа спецификации эконометрических моделей и две формы; обсуждение провели на примере простейшей модели спроса-предложения на конкурентом рынке.

3 эндогенные переменные: спрос, предложение и цена; 1 эндогенная: x. \\
На Семинаре 1 было обсуждено 2 принципа спецификации модели и 2 формы $(y^d, y^s, p), x$.

В этой модели (о взаимосвязи эндогенных  и экзогенных переменных) по существу заложено предположение, что эндогенные переменные реагируют на уровень душевого дохода ($x$) и уровень предложения мгновенно реагирует на цену блага ($p$).

Между тем уровень предложения блага ($y^d$) в текущем периоде обладает определённой инерцией по отношению к изменению цены блага ($p$). Точнее уровень предложения в текущем периоде лучше объясняется ценой блага в предшествующем периоде ($p_{t-1}$), так как производителю необходимо время для перестройки производства. Подчеркнём что в этом утверждении содержится фактор времени и мы обязаны в процессе записи математическим языком данного утверждения различать цену блага в текущем периоде ($p_t$) и в предшествующем ($p_{t-1}$).

Обозначим цену блага текущем периоде $p_t$, обозначим цену блага в предшествующем $p_{t-1}$ (лагавой ценой). Таким образом мы можем сформулировать закон: Уровень предложения растёт ($y_t^s = y_t^s(p_{t-1})\uparrow$) с ростом лаговой цены $p_{t-1} \uparrow$. Напротив уровень блага \underline{мгновенно} реагирует на изменение цены ($p_t$) и душевного дохода $x$: $y_t^d = y_t^d(p_t, x)\downarrow \uparrow$ (в таком случае благо нормальное и ценное).

Третий закон формирования рыночной цены в текущем периоде сохраняется и в данном случае:\textit{$p_t$ (цена в текущем периоде) формируется при балансе текущего спроса и текущего предложения}. Требутся составить модель которая позволяет объяснять уровень спроса ($y_t^s$), уровень предложения ($y_t^d$) душевым доход в текущем периоде ($p_t$).
\begin{equation*}
\begin{cases}
p_t, p_{t-1} \\
y_t^s = y_t^s(p_{t-1})\uparrow \\
y_t^d = y_t^d(p_t, x)\downarrow \uparrow
\end{cases}
\end{equation*}
Таким образом в данной задаче с уточнённым законом предложения будут присутствовать:
\begin{itemize}
\item Текущей эндогенной перменной
\begin{equation}
(y_t^s, y_t^d, p_t)
\end{equation}
\item Текущая экзогенная и лагавая эндогенная переменная
\begin{equation}
(x_t, p_{t - 1})
\end{equation}
\end{itemize}
\begin{equation}
\begin{cases}
\begin{cases}
y_t^d = a_0 + a_1 p_t + a_2  x_t - \text{(прос. лин. модель спроса)}, \\
a_1 < 0, a_2 > 0
\end{cases} \\
\begin{cases}
y_t^s = b_0 + b_1 p_{t - 1} - \text{(прос. лин. модель предложения)}, \\
b_0 > 0
\end{cases} \\
y_t^s = y_t^d
\end{cases}
\end{equation}

Три уравнения образуют структурную форму простейшей экономической модели нормально ценного блага на конкурентном рынке.

\textbf{Итог.} Для отражения в модели фактора времения все переменные модели датируются, т.е. привязываются ко времени и в итоге возникает спецификация динамической модели. Подчеркнём, что в набор объясняющих переменных (2) могут входить лаговые эндогенные переменные.

\framebox[1.1\width]{Д/з} $p_t^m$ или $p_{t-1}^m$

\textbf{Задача.} Трасформировать модель (3) к приведённой
$$a_0 + a_1 p_t + a_2 x_1 = b_0 + b_1 p_{t - 1}$$
$$a_1 p_t = b_0 + b_1 p_{t-1} - a_2 x_t - a_0$$
$$p_t = \frac{b_0 + b_1 p_{t-1} - a_2 x_t - a_0}{a_1}$$
$$p_t = \frac{b_0 - a_0}{a_1} + \frac{b_1}{a_1} p_{t-1} - \frac{a_2}{a_1} x_t$$
Приведённая форма предложения уже содержится в структурной форме (3):
$$y_t^s = b_0 + b_1 p_{t - 1}$$

В силу $y_t^s = y_t^d$, $y_t^s = b_0 + b_1 p_{t - 1} = y_t^d$

Получаем систему уравнений:
\begin{numcases}{}
	y_t^s = b_0 + b_1 p_{t - 1} = y_t^d; \\
	p_t = \frac{b_0 - a_0}{a_1} + \frac{b_1}{a_1} p_{t-1} - \frac{a_2}{a_1} x_t;
\end{numcases}

Это система называется простейшей динамической моделью спроса и предложения.

Сопоставляя приведённые формы статической модели спроса и предложения (Семинар 1) и динамической модели мы видим, что это совершенно различные модели.

\section*{Предельные величины в экономике}
Вернёмся к приведённой форме (5) и обозначим: $\alpha_0 = \displaystyle{\frac{b_0 - a_0}{a_1}}$, $\alpha_1 = \displaystyle{\frac{b_1}{a_1}}$, $\alpha_2 = \displaystyle{\frac{a_2}{a_1}}$ получим:
\begin{equation}
p = \alpha_0 + \alpha_1 p_{t-1} - \alpha_2 x_t
\end{equation}
Наша цель выяснить экономический смысл $\alpha_1, \alpha_2$. Предположим, что $p_{t - 1}, x_t + \triangle x_t$, тогда:
\begin{equation}
p_t + \triangle p_t = \alpha_0 + \alpha_1 p_{t-1} + \alpha_2(x_t + \triangle x_t)
\end{equation}
Вычитая из уравнения (7) - (6) получим:
\begin{equation}
\triangle p_t = \alpha_2 \cdot \triangle x_t
\end{equation}
Предположим, что $\triangle x_t = 1$, тогда $\triangle p_t = \alpha_2$

Таким образом $\alpha_2$ изменение эндогенной переменной $p_t$ в ответ на дополнительную еденицу, объясняющую $x_t$.

$\alpha_2$ -- предельным значением $p_t$ по объясняющей переменной $x_t$.

Добавим, что $\alpha_2$ можно расчитать по правилу:
\begin{equation}
\alpha_2 = \frac{\partial p_t}{\partial x_t}
\end{equation}

\textbf{Задача}. Вычислить $\alpha_2$ и дать экономическую интерпретацию.

Рассматривая знаки коэффициентов в структурной форме (3) и выражение коэффициента (5), мы убеждаемся, что $\alpha_2 > 0$.

\framebox[1.1\width]{Д/з} Уточнить динамический закон предложения, согласно уточнённому закону
$y_t^s = y_t^s(p_{t-1})\uparrow  \Rightarrow y_t^s = y_t^s(p_{t-1}, p_{t-1}^m)\uparrow \downarrow$.
Лагаваю цену сырья ($p_{t-1}^m$) интерпретировать, как лаговую экзогенную переменную. Трансформировать такую динамическую модель к приведённой форме. И выяснисть знак у текущего спроса по лаговой цене сырья $\displaystyle{\frac{\partial y_t^d}{\partial p_{t-1}^{(m)}}}$
\end{document}
