\documentclass[12pt,a4paper]{article}
\usepackage[14pt]{extsizes}
\usepackage[utf8]{inputenc}
\usepackage{amsmath}
\usepackage{amsfonts}
\usepackage{amssymb}
\usepackage{cmap}
% for fonts
    \usepackage[T2A, T1]{fontenc}
    \usepackage[english, russian]{babel}
    \usepackage{fontspec}
    \defaultfontfeatures{Ligatures=TeX,Renderer=Basic}
    \setmainfont[Ligatures={TeX, Historic}]{Times New Roman}
    \setsansfont{Times New Roman}
    \setmonofont{Courier New}
% mathcha
\usepackage{tikz}
\usepackage{mathdots}
\usepackage{yhmath}
\usepackage{cancel}
\usepackage{color}
\usepackage{siunitx}
\usepackage{array}
\usepackage{multirow}
\usepackage{amssymb}
\usepackage{gensymb}
\usepackage{tabularx}
\usepackage{booktabs}
\usetikzlibrary{fadings}
% mathcha
\usepackage{pgfplots} % plot
\usepackage{float} % for H at figure
\usepackage{cases}
\pgfplotsset{compat=1.15}
\usepackage{graphicx}
\usepackage[left=2cm,right=2cm,top=2cm,bottom=2cm]{geometry}
\author{Аверьянов Тимофей, Корякин Алексей}
\begin{document}
\begin{center}
\section*{Семинар №5 \\
Линейная модель множественной регрессии и оценивание её параметров при помощи функции line}
\end{center}
\begin{enumerate}
\item Проверка адекватности модели Самуэльсона-Хигса для гос. Расходов России
\item Модифицированная модель Самуэльсона-Хигса расходов домохозяйств в России и её оценивание при помощи функции line excel
\item Проверка ДЗ
\end{enumerate}

На занятии вычислили значения случайных возмущений по правилу:
\begin{equation*}
\vec{w}_{t} \ =\ G_{t} \ -\tilde{g} \ G_{t-1}
\eqno(4)
\end{equation*}
\begin{equation*}
\tilde{\sigma }_{w} \ =\ \sqrt{\frac{{\displaystyle \sum \tilde{w}^{2}_{t}}}{n\ -\ k}} \approx 140\
\eqno(5)
\end{equation*}
	Таким образом \textbf{третий этап} схемы завершается записью оценнённой модели:


\begin{equation*}
\begin{cases}
G_{t} \ =\ 1.008\cdot \tilde{g} \cdot G_{t-1} \ +\ w_{t}\\
\tilde{\sigma }_{w} =140
\end{cases}
\eqno(6)
\end{equation*}
	\textbf{На четвёртом этапе} осуществляется прогноз по оценённой модели значений эндогенных переменных из контролирующей выборки. В нашем примере рассчитывается расход на 2018 год. После рассчёта прогноза вычисляется относительная ошибка прогноза.
\begin{equation*}
\tilde{G}_{2018} =\tilde{g} \ ( =\ 1.008) \cdot G_{2017}
\eqno(7)
\end{equation*}
	Модель является адекватной, если относительная ошибка прогноза не превосходит 15\% от прогнозируемых велечин.
\begin{equation*}
\delta \ =\ 100\ \cdot \ |\tilde{G}_{2018} \ -\ G_{2018} |\ \leq \ 15\%
\eqno(8)
\end{equation*}
\textbf{Задача.}

	Вычислить по правилу (7) прогноз и проверить адекватность модели:
\begin{equation*}
\tilde{G}_{2018} \ =\ \tilde{g} \ \cdot G_{2017} \ =\ 1.008\ \cdot 7264.2719268\ =7322.386
\end{equation*}
\begin{equation*}
\delta \ =\ 100\ \cdot \ |7320\ -\ 7322.386|\ =0.09\%
\end{equation*}
	Ещё один вариант суждения об адекватности модели базируется на правиле 2-3 сигм ($\displaystyle \sigma $): модель признаётся адекватной, если абсолютные ошибки прогноща не превосходя 2-3 $\displaystyle \sigma $ в нашем примере мы бы признали модель (6) адекватной, если абсолютная ошибка прогноза:
\begin{equation*}
e\left(\tilde{G}_{2018}\right) \ =\ |\tilde{G}_{2018} \ -\ G_{2018} |\ \leq 2\ \cdot \tilde{\sigma }_{w} \ =\ 280\ \text{млрд. руб.}
\end{equation*}
\textbf{	Модифицируем модель Самуэльсона-Хикса} при помощи более глубоко обсуждения диаграммы рассеивания. Вернёмся к диаграмме рассеивания "лаговое ВВП России - текущее потребление домохозяйтсв" и внимательно изучим эту диаграмму, обращая внимание на наличие явныъ выбросов. Рассматривая диаграмму можем сделать следующий вывод: первый очевидный выброс датируется 2009 годом $\displaystyle C$ это значит, что его причиной является мировой финансвый кризис. Остальные выбросы датируется следующими годами: 2015, 2016, 2017. И их причиной являются санкции западных стран.

\textbf{	Итог: }модель Самуэльсона-Хикса нужно модифицировать отразив в ней воздействие мирового крищиса и санкции западных стран. Вот модифиццированных фрагметн модели Самуэльсона-Хикса рассхода домохозяйств России.

Обозначим фиктивную переменную свзязанную с мировым финансовым кризисом $\displaystyle Gr_{t}$ она равна 0, если в период $\displaystyle t$ кризис отсутсвует и 1, если существует


\begin{equation*}
Gr_{t} \ =\begin{cases}
\ 0,\ \text{если в период t кризис отсутсвует}\\
1,\ \text{если существует}
\end{cases}
\eqno(9)
\end{equation*}
$\displaystyle Gr_{t} \ -$это \textit{индикатор кризиса}. Аналогично индикатор крищиса:


\begin{equation*}
San_{t} \ =\ \begin{cases}
1\ ,\text{если существует}\\
0,\ \text{если отсутвует}
\end{cases}
\eqno(9)
\end{equation*}
	С помощью этих величин модифицируем модель Самуэльсона-Хикса:


\begin{equation*}
\begin{cases}
C_{t} =\ a_{0} \ +\ a_{1} \ \cdot Y_{t-1} \ +\ a\ _{2} \ Gr_{t} \ +\ a_{3} \ \cdot San_{t} \ +\ u_{t}\\
E( u_{t}) \ =\ 0;\ Var( u_{t}) \ =\ \sigma ^{2}
\end{cases} \
\eqno(10)
\end{equation*}
	Спецификация (10) включает в себя 5 параметров: $\displaystyle ( a_{0} ,\ a_{1} ,a_{2} ,\ a_{3} ,\ \sigma _{u})( 11)$. Обратим внимание, что спецификация (10) служит конкретным примером базовой модели эконометрики, которая носит называние \textit{линейной модели множественной регресии}. Добавим, что при определённых свойствах случайного возмущения $\displaystyle u_{t}$ параметры модели (11) оптимально оцениваются методом наименьших квадратов и на сегодняшнем занятии мы познакомимся с функцией ЛИНЕЙН в которой запрограммирована процедура наименьших квадратов с которой мы познакомились на прошлом занятии при оценивании модели (1).
\begin{center}
\textbf{Оценивание параметров (11) при помощи функции ЛИНЕЙН}
\end{center}

\begin{enumerate}
\item На листе Excel занесём сивмол $\displaystyle date$ времени $\displaystyle t$(2003 -2017). Символом $\displaystyle C_{t}$ - ВВП, $\displaystyle Y_{t-1}$ - лаговый доход. $\displaystyle Cr_{t} \ -$ кризис. $\displaystyle San_{t} \ -$ санкции. $\displaystyle n=15$. Ввели заголовки и заполнили значениями переменных из обучающей модели. \textit{Замечание. }Результат первого шага можно интерпретировать, как запись уравнений наблюдений в рамках модели (10)
\item Размещаем курсор со значениями эндогенных переменных ($\displaystyle C_{t}$) и кликаем по символу формул. В стоблце категория выбираем статистические далее выбираем ЛИНЕЙН и кликаем OK. В первую строчку заносим адрес массива эндогенной переменной. Во вторую строчку заносим объясняющие. В третьей и четрвертой строчке следует вывеости 1.
\item Запишем модель точно также как модель (1).
\end{enumerate}


\begin{equation*}
\begin{cases}
C_{t} \ =\ -12107\ +\ 0.84\ Y_{t-1} \ -\ 3606\ Gr_{t} \ -\ 2009\ \cdot San_{t} \ +\ u_{t}\\
\tilde{\sigma }_{w} \ =406.5\ м.р.
\end{cases} \
\eqno(12)
\end{equation*}
\textit{	Комментарий}. В первой строчке выделенного массива (протокола функции линейн) расположенны в обратном порядке оценки коэффициентов. Мера точности во второй строке. Величина $\displaystyle \tilde{\sigma }_{w}$ всегда содердится в 3 строчке 2 столбца протокола. Остальное содержимое мы обсудим позже.

$\displaystyle \boxed{\text{ДЗ}}$Проанализировать диаграмму рассеивания остальных элементов диаграммы Самуэльсона-Хикса и если есть основания модифицировать отразив на низ влияние кризиса и санкций. Воспользоваться функцией линейн и оценить параматеры двух остальных параметров модели Самуэльсона-Хикса.
\end{document}
